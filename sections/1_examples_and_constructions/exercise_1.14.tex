\subsection{Exercise~1.14}
\label{exercise 1.14}



\subsubsection{a)}

\begin{proposition}
	\label{projections are open}
	Let~$(X_α)_α$ be a family of topological spaces, and for every index~$β$ let~$π_β$ be the canonical projection map from~$∏_{α ∈ A} X_α$ to~$X_β$.
	Each map~$π_β$ is open.
\end{proposition}

\begin{proof}
	Openness of a map can be checked on a basis.
	A basis for the product topology is given by the sets~$∏_{α ∈ A} U_α$ with~$U_α = X_α$ for all but finitely many indices, and each~$U_α$ an open subset of~$X_α$.
	Applying the projection~$π_β$ to such a basis element gives either the set~$U_β$ or~$∅$ (if some factor~$X_α$ with~$α ≠ β$ is empty).
	In either case, we get an open subset of~$X_β$.
\end{proof}

It follows from the above \lcnamecref{projections are open} that the projection map
\[
	π \colon ℝ^2 \to ℝ \,, \quad (x, y) \mapsto x
\]
is open.
But this map is not closed:
the set
\[
	C ≔ \{ (x, y) ∈ ℝ^2 \suchthat x y = 1 \}
\]
is closed in~$ℝ^2$, but the set~$π C = ℝ ∖ \{ 0 \}$ is not closed in~$ℝ$.



\subsubsection{b), First Solution}

We consider the continuous surjection
\[
	f \colon [0, 2\pi] \to \sphere^1 \,, \quad t \mapsto \eul^{it} \,.
\]
The closed interval~$[0, 2\pi]$ is compact, so every closed subspace~$C$ of~$[0, 2\pi]$ is again compact.
It follows that~$f C$ is compact, whence~$f C$ is closed in~$\sphere^1$ (since~$\sphere^1$ is a Hausdorff space).
This shows that~$f$ is closed.

But~$f$ is not open:
the set~$[0, \pi)$ is open in~$[0, 2\pi]$, but its image under~$f$ is not open in~$\sphere^1$.


\subsubsection{b), Second Solution}

We construct an example of a continuous surjection~$f \colon X \to S$ that is closed but not open, and where both~$X$ and~$S$ are finite.
To this end, we make the following observations:
\begin{itemize*}

	\item
		If the topology on~$S$ is the discrete one, then~$f$ will automatically be both open and closed.
		We must therefore choose~$S$ non-discrete.

		Similarly, if the topology on~$X$ is the indiscrete one, then~$f$ will automatically be both open and closed.
		We must therefore choose~$X$ as non-indiscrete.


	\item
		The space~$X$ must have at least as many points as the space~$S$ for~$f$ to be surjective.

		Suppose for a moment that~$X$ and~$S$ had the same number of points.
		The continuous surjection~$f$ would then be a continuous bijection.
		It would then follow from the upcoming part~c) that~$f$ would already be a homeomorphism.
		This would entail that~$f$ is also open.

		The space~$X$ must therefore have strictly more points than the space~$S$.

\end{itemize*}

We choose~$S$ as the smallest possible non-discrete topological space:
the two-point indiscrete space~$S = \{ a, b \}$.
We also choose~$X$ as the smallest possible non-indiscrete topological space with strictly more than two elements:
the three-point space with set~$X = \{ x, y, z \}$ and topology~$\top{T}_X = \{ ∅, \{ x \}, X \}$.
The function
\[
	f \colon X \to S \,, \quad x, y \mapsto a \,, \quad z \mapsto b
\]
is surjective, and it is continuous because the topology on~$S$ is indiscrete.
The closed subsets of~$X$ are~$∅$,~$\{ y, z \}$ and~$X$, each of which has closed image in~$S$.
Therefore,~$f$ is closed.
But~$f$ is not open because~$\{ x \}$ is open in~$X$ while~$f \{ x \} = \{ a \}$ is not.
See~\cref{continuous surjection between finite sets} for a picture of~$f$.
\begin{figure}
	\centering
	\begin{tikzpicture}[scale = 0.5]
		% elements
		\node (x) at (0, 4) {$x$};
		\node (y) at (0, 2) {$y$};
		\node (z) at (0, 0) {$z$};
		\node (a) at (5, 3) {$a$};
		\node (b) at (5, 1) {$b$};
		% open sets for x, y, z
		\draw[black!70] (x) circle (0.8);
		\draw[black!70] (y) ellipse (1.7 and 3.4);
		% open sets for a, b
		\draw[black!70] ($(a)!0.5!(b)$) ellipse (1 and 2);
		% draw lines
		\draw (x) -- (a);
		\draw (y) -- (a);
		\draw (z) -- (b);
	\end{tikzpicture}
	\caption{A continuous surjection beween finite topological spaces that is closed but not open.}
	\label{continuous surjection between finite sets}
\end{figure}

%We choose~$S$ as the smallest non-discrete, non-indiscrete space:
%the Sierpiński space~$S = \{ a, b \}$ whose topology is given by~$\top{T}_S = \{ ∅, \{ a \}, S \}$.
%As seen above, the desired space~$X$ need to have at least three elements.
%We make some further observations regarding continuous maps from~$X$ to~$S$.
%\begin{itemize*}
%
%	\item
%		A continuous map~$f$ from~$X$ to the Sierpiński space~$S$ corresponds to the open subset~$U ≔ f^{-1} \{a\}$ of~$X$.
%
%	\item
%		For~$f$ to be surjective, both the open set~$U$ and the closed set~$X ∖ U$ must be nonempty.
%
%	\item
%		The only non-open subset of~$S$ is~$\{ b \}$.
%		For~$f$ not to be open, there hence must exist a nonempty open subset~$V$ of~$X$ contained in~$X ∖ U$.
%
%	\item
%		The only non-closed subset of~$S$ is~$\{ a \}$.
%		For~$f$ to be closed, there hence must not exist a nonempty closed subset~$C$ of~$X$ with~$C ⊆ U$.
%		Equivalently, there must not exist a proper open subset~$W$ of~$X$ with~$X ∖ U ⊆ W$.
%
%\end{itemize*}
%
%In other words:
%we need a topological space~$X$ together with two nonempty open subsets~$U$ and~$V$ such that
%\begin{itemize*}
%
%	\item
%		$U$ and~$V$ are disjoint,
%
%	\item
%		the only open subset~$W$ of~$X$ with~$X ∖ U ⊆ W$ is~$W = X$.
%
%\end{itemize*}
%
%If we are given such a space~$X$, then we can identify all points in~$U$ to get a smaller space~$X'$ that also satisfies all these properties.
%We can then further identify all points in~$V$ to get an even smaller space~$X''$ that once again satisfies all these properties.
%We may therefore assume that both~$U$ and~$V$ consist of only a single point, say~$U = \{ x \}$ and~$V = \{ y \}$.
%The second condition, that the only open subset of~$X$ that contains~$X ∖ \{ x \}$ is all of~$X$, is then equivalent to~$X ∖ \{ x \}$ not being open.
%We can ensure this by simply adding next to the two open points~$\{ x \}$ and~$\{ y \}$ another point~$z$ such that~$\{ y, z \}$ is \emph{not} open in~$X$;
%this also entails that~$\{ z \}$ cannot be open.
%
%We arrive overall at the topological space
%\[
%	X ≔ \{ x, y, z \} \,,
%	\quad
%	\top{T}_X ≔ \{ ∅, \{ x \}, \{ y \}, \{ x, y \}, X \} \,,
%\]
%and the surjective map
%\[
%	f \colon X \to S \,, \quad x \mapsto a \,, \quad y, z \mapsto b \,,
%\]
%see \cref{continuous surjection between finite sets}.
%\begin{figure}
%	\centering
%	\begin{tikzpicture}[scale = 0.75]
%		% elements
%		\node (x) at (0, 4) {$x$};
%		\node (y) at (0, 2) {$y$};
%		\node (z) at (0, 0) {$z$};
%		\node (a) at (5, 3) {$a$};
%		\node (b) at (5, 1) {$b$};
%		% open sets for x, y, z
%		\draw[black!70] (x) circle (0.5);
%		\draw[black!70] (y) circle (0.5);
%		\draw[black!70] ($(x)!0.5!(y)$) ellipse (1.1 and 1.9);
%		\draw[black!70] ([shift={(0, 0.2)}]y) ellipse (1.6 and 3.2);
%		% open sets for a, b
%		\draw[black!70] (a) circle (0.5);
%		\draw[black!70] ($(a)!0.5!(b)$) ellipse (1 and 2);
%		% draw lines
%		\draw (x) -- (a);
%		\draw (y) -- (b);
%		\draw (z) -- (b);
%	\end{tikzpicture}
%	\caption{A continuous surjection that is closed but not open.}
%	\label{continuous surjection between finite sets}
%\end{figure}
%The map~$f$ is continuous since it corresponds to the open subset~$\{ x \}$ of~$X$.
%There is only one non-closed subset in~$S$, namely~$\{ a \}$, and only one subset of~$X$ that is mapped to~$\{ a \}$, namely~$\{ x \}$.
%Because~$\{ x \}$ is not closed in~$X$, we find that~$f$ is closed.
%But~$f$ is not open, since~$\{ y \}$ is an open subset of~$X$ for which~$f \{ y \} = \{ b \}$ is not open in~$S$.



\subsubsection{c)}

For every open subset~$U$ of~$Y$, its preimage~$f^{-1} U$ is open in~$X$ by the continuity of~$f$.
Suppose conversely that~$f^{-1} U$ is open in~$X$.

If~$f$ is open, then it follows from the equality~$U = f f^{-1} U$ (which holds because~$f$ is surjective) that~$U$ is open in~$Y$.

Suppose otherwise that~$f$ is closed.
For the set~$C ≔ Y ∖ U$ its preimage~$f^{-1} C = X ∖ f^{-1} U$ is closed in~$X$, so~$f f^{-1} C = C$ is closed in~$Y$.
Therefore,~$U = Y ∖ C$ is open in~$Y$.
