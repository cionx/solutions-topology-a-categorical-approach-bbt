\subsection{Example~1.5}

We note that the proposed equation~$ℤ ∖ \{-1, 1\} = ⋃_{\text{$p$ prime}} S(p, 0)$ cannot be true with the given definition of~$S(a, b)$, since the right-hand side consists no negative integers.
Comparing with other source seems to suggest that the definition of~$S$ should use~$ℤ$ instead of~$ℕ$, so that
\[
	S(a, b) = \{ a n + b \suchthat n ∈ ℤ \} = a ℤ + b \,.
\]
In the following, we will use this definition instead.

The book isn’t quite clear on how the given topology is supposed to be defined.
\begin{itemize}

	\item
		The given formulation seems to suggest that the set
		\[
			\{ S(a, b) \suchthat a, b ∈ ℤ, a ≠ 0 \} ∪ \{ ∅ \}
		\]
		is supposed to be a topology on~$ℤ$.
		But this would be false, since the union~$S(2, 0) ∪ S(3, 0)$ would not be open again.

	\item
		One could also read the given formulation as saying that there exists \emph{some} topology on~$ℤ$ for which these sets are open.
		But this seems strange for at least two reasons:
		\begin{itemize}

			\item
				Why explicitly mention that~$∅$ is supposed to be open?
				Every topology automatically satisfies this condition.

			\item
				The existence of such a topology would be trivial, since the discrete topology would do the job.

		\end{itemize}
		The later statement that \enquote{no nonempty finite set can be open} would also not necessarily be true.
\end{itemize}

Instead, the proper claim should probably be that the set
\[
	\{ S(a, b) \suchthat a, b ∈ ℤ, a ≠ 0 \}
\]
is a basis for a topology on~$ℤ$.
We prove this claim by showing the following more general version:

\begin{proposition}
	\label{topolgies from subgroups}
	Let~$A$ be an abelian group.
	Let~$\mathcal{S}$ be a nonempty collection of subgroups of~$A$ such that for every two subgroups~$S_1$ and~$S_2$ contained in~$\mathcal{S}$, there exists a third subgroup~$T$ contained in~$\mathcal{S}$ with~$T ⊆ S_1 ∩ S_2$.
	Then the set
	\[
		\basis{B} ≔ \{ a + S \suchthat S ∈ \mathcal{S} \}
	\]
	is a basis for a topology on~$A$.
	In this topology, the sets belonging to~$\basis{B}$ are not only open but also closed.
\end{proposition}

\begin{proof}
	By assumption, there exists an element~$S$ of~$\mathcal{S}$.
	Every element~$a$ of~$A$ is then contained in the corresponding set~$a + S$ belonging to~$\basis{B}$.
	This shows that~$\basis{B}$ satisfies the first property of a basis.

	Let~$B$ and~$C$ be two elements of~$\basis{B}$ and let~$x$ be an element of~$B ∩ C$.
	There exists elements~$S_1$ and~$S_2$ of~$\basis{B}$ and elements~$a_1$ and~$a_2$ of~$A$ with~$B = a_1 + S_1$ and~$C = a_2 + S_2$.
	The sets~$a_1 + S_2$ and~$a_2 + S_2$ are cosets of~$S_1$ and~$S_2$ respectively, which contain~$x$.
	We have therefore~$a_1 + S_1 = x + S_1$ and also~$a_2 + S_2 = x + S_2$, and thus~$B = x + S_1$ and~$C = x + S_2$.
	It follows that
	\[
		B ∩ C
		=
		(x + S_1) ∩ (x + S_2)
		=
		x + (S_1 ∩ S_2) \,.
	\]
	(We note that the map~$A \to A$ given by~$a \mapsto x + a$ is bijective, and therefore preserves intersections.)
	There exists by assumption some element~$T$ of~$\mathcal{S}$ with~$T ⊆ S_1 ∩ S_2$.
	We have~$x ∈ x + T$ and~$x + T ⊆ B ∩ C$, with~$x + T$ belonging to~$\basis{B}$.
	This shows that~$\basis{B}$ satisfies the second property of a basis.

	Let~$B$ be a set belonging to~$\basis{B}$.
	This set is of the form~$B = a + S$ for some~$a ∈ A$ and a subgroup~$S$ belonging to~$\mathcal{S}$.
	The complement of~$B$ is the union of all other cosets of~$S$, all of which also belong to~$\basis{B}$ and are therefore open.
	The complement of~$B$ is therefore open, whence~$B$ is closed.
\end{proof}

The above \lcnamecref{topolgies from subgroups} has the following special cases:
\begin{itemize*}

	\item
		For every integral domain~$R$ we can choose~$\mathcal{S}$ as the set of nonzero ideals of~$R$: for every two ideals~$I$ and~$J$ of~$R$, the product~$IJ$ is again nonzero.
		(The intersection~$I ∩ J$ is therefore also nonzero, as it contains~$IJ$.)

	\item
		For every ring~$R$ and every ideal~$I$ we can choose~$\mathcal{S}$ is the set of ideals~$I^k$ with~$k ≥ 0$.
		(The resulting topology on~$R$ is the~$I$\nobreakdash-adic topology.)

\end{itemize*}

We have thus shown that there exists a topology on~$ℤ$ with basis given by~$\{ S(a, b) \suchthat a, b ∈ ℤ, a ≠ 0 \}$, and each of these basis open sets is also closed.
Every basis open set is infinite, whence every nonempty open set is infinite.
The set~$ℤ ∖ \{-1, 1\}$ is therefore not closed.
But the set~$S(p, 0)$ is closed for each prime~$p$, and the union of finitely many closed sets is again closed.
We thus find from the equality~$ℤ ∖ \{-1, 1\} = ⋃_{\text{$p$ prime}} S(p, 0)$ that there are infinitely many primes.
