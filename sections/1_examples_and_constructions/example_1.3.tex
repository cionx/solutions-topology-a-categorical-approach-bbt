\subsection{Example~1.3}

We check that the collection of half-open intervals~$\basis{B} = \{ [a, b) \mid a, b ∈ ℝ, a < b \}$ is indeed a basis for a topology on~$ℝ$.

Every point~$x$ in~$ℝ$ is contained in the element~$[x, x+1)$ of~$\basis{B}$.
This shows that~$\basis{B}$ satisfies the first property of a basis.

Let~$B$ and~$C$ be two sets belonging to~$\basis{B}$ and let~$x$ be a point in the intersection~$B ∩ C$.
The sets~$B$ and~$C$ are of the form~$B = [a, b)$ and~$C = [c, d)$ for real numbers~$a$,~$b$,~$c$ and~$d$ with~$a < b$ and~$c < d$.
The intersection~$B ∩ C = [a, b) ∩ [c, d)$ is given by~$[e, f)$ for the numbers~$e = \max(a, c)$ and~$f = \min(b, d)$.
We must have~$e < f$ since~$[e, f)$ is nonempty, as~$x$ is contained in~$[e, f)$.
Therefore,~$[e, f)$ belongs to~$\basis{B}$, and we have~$x ∈ [e, f) ⊆ B ∩ C$
This shows that~$\basis{B}$ satisfies the second property of a basis.
