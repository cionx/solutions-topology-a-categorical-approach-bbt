\subsection{Exercise~1.8}
\label{exercise 1.8}



\subsubsection{Universal Property, First Solution}

We show that the better definition of the product topology agrees with the classic definition, as provided in Definition~1.4.
Let~$\top{T}$ be the coarsest topology on~$X$ for which all the projections~$π_α$ are continuous.

For every open subset~$U$ of~$X_β$, its preimage~$π_β^{-1} U$ equals the set~$∏_{α ∈ A} U_α$ with~$U_β = U$ and $U_α = X$ for every~$α ≠ β$.
These preimages are contained in the given basis for the product topology, and are therefore open in the product topology.
This shows that with respect to the product topology all projections~$π_β$ are continuous.
Consequently,~$\top{T}$ is coarser than the product topology.

Let~$V = ∏_{α ∈ A} U_α$ be one of the given basis elements of the product topology.
This means that for finitely many distinct indices~$α_1, \dotsc, α_n$ the set~$U_{α_i}$ is some open subset of~$X_α$, and for every index~$β$ with~$β ≠ α_1, \dotsc, α_n$ the set~$U_β$ is equals~$X_β$.
This means that
\[
	V = π_{α_1}^{-1} U_{α_1} ∩ \dotsb ∩ π_{α_n}^{-1} U_{α_n} \,.
\]
Each preimage~$π_{α_i}^{-1} U_{α_i}$ is an element of~$\top{T}$, so it follows that~$V$ is also an element of~$\top{T}$.

We have thus found that every set of the basis of the product topology is contained in~$\top{T}$.
Consequently, the product topology in coarser than~$\top{T}$.



\subsubsection{Universal Property, Second Solution}

Let~$\top{T}$ denote the coarsest topology on~$X$ for which all projections~$π_α$ are continuous, and for every~$α ∈ A$ let~$\top{T}_α$ denote the topology of~$X_α$.
Using the notations introduced in \cref{pullback and pushforward of topologies} and \cref{notation for infima and suprema of topologies}, we find that
\[
	\top{T} = ⋁_{α ∈ A} π_α^* \top{T}_α \,.
\]
The topology~$\top{T}$ is thus generated by the set~$⋃_{α ∈ A} π_α^* \top{T}_α$.
According to \nameref{exercise 0.1}, the topology~$\top{T}$ has therefore a basis~$\basis{B}$ given all finite intersections of sets belonging to~$⋃_{α ∈ A} π_α^* \top{T}_α$.
In other words, the basis~$\basis{B}$ is given by the sets
\begin{equation}
	\label{formula for basis of product topology}
	π_{α_1}^{-1} U_1 ∩ \dotsb ∩ π_{α_n}^{-1} U_n
\end{equation}
with~$n ≥ 0$,~$α_1, \dotsc, α_n ∈ A$, and~$U_i ∈ \top{T}_{α_i}$ for every~$i = 1, \dotsc, n$.
Each of the sets~$π_α^* \top{T}_α$ is closed under finite intersections, as it is a topology on~$X$.
Consequently,~$\basis{B}$ is already given for all those sets of the form~\eqref{formula for basis of product topology} for which the indices~$α_1, \dotsc, α_n$ are pairwise distinct.
This is precisely the basis for the product topology given in its classical definition (Definition~1.4).



\subsubsection{Uniqueness of Product Topology}

\begin{proposition}
	\label{topology uniquely determined by ingoing maps}
	Let~$X$ be a set and let~$\top{T}_1$ and~$\top{T}_2$ be two topologies on~$X$.
	Suppose that for every topological space~$Z$ and every map~$f$ from~$Z$ to~$X$, the map~$f$ is continuous with respect to~$\top{T}_1$ if and only if it is continuous with respect to~$\top{T}_2$.
	Then~$\top{T}_1 = \top{T}_2$.
\end{proposition}

\begin{proof}
	The identity map~$\id_X$ is continuous from~$\top{T}_1$ to~$\top{T}_1$, and therefore also from~$\top{T}_1$ to~$\top{T}_2$.
	This means that~$\top{T}_1$ is finer than~$\top{T}_2$.
	By swapping around the roles of~$\top{T}_1$ and~$\top{T}_2$, we also find that~$\top{T}_2$ is finer than~$\top{T}_1$.
\end{proof}

The given assertion is a direct consequence of the above \lcnamecref{topology uniquely determined by ingoing maps}.
