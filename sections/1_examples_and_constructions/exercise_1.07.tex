\subsection{Exercise~1.7}

\begin{definition}
	Let~$X$ be a topological space, let~$(x_n)_n$ be a sequence in~$X$ and let~$x$ be a point in~$X$.
	The sequence~$(x_n)_n$ \defemph{converges} to~$x$, denoted by~$x_n \to x$ as~$n \to ∞$, if for every open neighbourhood~$U$ of~$x$ there exists some~$N$ such that~$x_n ∈ U$ for every~$n ≥ N$.
\end{definition}

In metric spaces, and therefore also in normed vector spaces, the above definition of convergence of sequences agrees with the usual one.

We consider in the vector space~$\Cont([0, 1])$ the sequence~$(f_n)_n$ given by
\[
	f_n(x) = x^n
\]
for all~$x ∈ [0, 1]$,~$n ≥ 0$.
We have
\[
	\norm{f_n}_1
	=
	∫_0^1 \abs{f_n}
	=
	∫_0^1 x^n \dd{x}
	=
	\frac{1}{n + 1}
	\to
	0
\]
as~$n \to ∞$, and therefore~$f_n \to 0$ as~$n \to ∞$ with respect to the topology induced by~$\norm{\phdot}_1$.
But we have
\[
	\norm{f_n}_∞ = 1
\]
for every~$n ≥ 0$, so the sequence~$(f_n)_n$ does not converge towards~$0$ with respect to the topology induced by~$\norm{\phdot}_∞$.

This shows that the topologies induced by~$\norm{\phdot}_1$ and~$\norm{\phdot}_∞$ cannot be the same.
