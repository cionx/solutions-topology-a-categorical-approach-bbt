\subsection{Example~1.4}

We check that the given collection of sets
\[
	\basis{B}
	= \{ (a, b) \suchthat a, b ∈ X, a < b \}
		∪ \{ (a, ∞) \suchthat a ∈ X \}
	  ∪ \{ (-∞, b) \suchthat b ∈ X \}
\]
is a topology for a basis on~$X$.

We first note that this statement is wrong if~$X$ is a one-element set.
We also observe that the assertion is true if~$X$ is empty (since for the empty topological space, every collection of subsets is a basis).
We will therefore assume in the following that~$X$ contains at least two different elements.

Let~$x$ be an element of~$X$.
There exist by assumption an element~$y$ of~$X$ with~$y ≠ x$, and thus either~$x < y$ or~$y < x$.
It follows that~$x$ is contained in~$(-∞, y)$ or in~$(y, ∞)$, with both of these sets belonging to~$\basis{B}$.
This shows that~$\basis{B}$ satisfies the first property of a basis.

To show that~$\basis{B}$ satisfies the second property of a basis we may extend~$\basis{B}$ to~$\basis{B}' = \basis{B} ∪ \{ X \}$, and show that~$\basis{B}$ satisfies the second property of a basis.
We also set~$X' ≔ X ∪ \{ ∞, -∞ \}$, so that
\[
	\basis{B}' = \{ (a, b) \suchthat a, b ∈ X' \} \,.
\]

Let~$B$ and~$C$ be two sets belonging to~$\basis{B}'$.
There exist elements~$a, b, c, d ∈ X'$ with~$B = (a, b)$ and~$C = (c, d)$.
The intersection~$B ∩ C$ is thus given by~$(e, f)$ with~$e = \min(a, c)$ and~$f = \max(b, d)$.
If~$x$ is a point in~$B ∩ C$ then we must have~$e < f$ whence~$B ∩ C = (e, f)$ again belongs to~$\basis{B}'$.
It follows that~$\basis{B}'$ satisfies the second property of a basis.
