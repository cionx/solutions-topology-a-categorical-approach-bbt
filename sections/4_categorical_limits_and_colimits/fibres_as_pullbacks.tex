\subsection{Fibres as Pullbacks}

We consider the following diagram of sets and maps between them:
\[
	\begin{tikzcd}
		{}
		&
		\ast
		\arrow{d}[right]{f}
		\\
		Y
		\arrow{r}[above]{g}
		&
		Z
	\end{tikzcd}
\]
Let~$i$ be the inclusion map from the fibre~$g^{-1} z$ into~$Y$.
The resulting diagram
\[
	\begin{tikzcd}
		g^{-1} z
		\arrow{r}
		\arrow{d}[left]{i}
		&
		\ast
		\arrow{d}[right]{f}
		\\
		Y
		\arrow{r}[above]{g}
		&
		Z
	\end{tikzcd}
\]
commutes.
Let~$W$ be another set and let~$h_1 \colon W \to \ast$ and~$h_2 \colon W \to Y$ be two maps such that the diagram
\[
	\begin{tikzcd}
		W
		\arrow{r}[above]{h_1}
		\arrow{d}[left]{h_2}
		&
		\ast
		\arrow{d}[right]{f}
		\\
		Y
		\arrow{r}[above]{g}
		&
		Z
	\end{tikzcd}
\]
commutes.
The commutativity of this diagram means that the composite~$g h_2$ is constant with value~$z$.
There hence exists a unique map~$h$ from~$W$ into the fibre~$g^{-1} z$ with~$h_2 = i h$.
This map~$h$ makes the overall diagram
\[
	\begin{tikzcd}
		W
		\arrow[bend left]{drr}[above right]{h_1}
		\arrow[bend right]{ddr}[below left]{h_2}
		\arrow{dr}[above right]{h}
		&[-1em]
		{}
		&
		{}
		\\
		{}
		&
		g^{-1} z
		\arrow{r}
		\arrow{d}[left]{i}
		&
		\ast
		\arrow{d}[right]{f}
		\\
		{}
		&
		Y
		\arrow{r}[above]{g}
		&
		Z
	\end{tikzcd}
\]
commute.
This shows that the fibre~$g^{-1} z$ together with the map~$g^{-1} z \to \ast$ and the inclusion map~$i \colon g^{-1} z \to Y$ is a pullback of the two maps~$f$ and~$g$.
