\subsection{Example~4.1}

We consider the diagram
\begin{equation}
	\label{direct limit diagram}
	\begin{tikzcd}[sep = normal]
		ℝ
		\arrow{r}[above]{f_1}
		&
		ℝ^2
		\arrow{r}[above]{f_2}
		&
		ℝ^3
		\arrow{r}[above]{f_3}
		&
		ℝ^4
		\arrow{r}
		&
		\dotsb
	\end{tikzcd}
\end{equation}
where for every~$n ≥ 1$ the map~$f_n$ is given by
\[
	f_n (x_1, \dotsc, x_n) = (x_1, \dotsc, x_n, 0) \,.
\]
For every~$k ≥ 1$ let~$ι_k$ be the inclusion map from~$ℝ^k$ into the direct sum
\[
	⨁_{n ≥ 1} ℝ
	=
	\{
		(x_1, \dotsc, x_n, 0, 0, \dotsc)
		\suchthat
		n ≥ 0,
		x_1, \dotsc, x_n ∈ ℝ
	\} \,,
\]
given by
\[
	ι_k (x_1, \dotsc, x_k) = (x_1, \dotsc, x_k, 0, 0, \dotsc) \,.
\]



\subsubsection{On the Level of Sets}

The maps~$ι_n$ with~$n ≥ 1$ satisfy the compatibility condition
\[
	ι_{n + 1} f_n = ι_n
\]
for every~$n ≥ 1$.
The set~$⨁_{n ≥ 1} ℝ$ together with the maps~$ι_n$ is therefore a cocone on the diagram~\eqref{direct limit diagram} in~$\Set$.


For every~$n ≥ 1$ let~$R_n$ be the image of the map~$ι_n$, that is,
\[
	R_n
	≔ \im ι_n
	= \{ (x_1, \dotsc, x_n, 0, 0, \dotsc) \suchthat x_1, \dotsc, x_n ∈ ℝ \} \,.
\]
The sets~$R_n$ form an increasing filtration of the set~$⨁_{n ≥ 1} ℝ$:
we have
\[
	R_1 ⊆ R_2 ⊆ R_3 ⊆ \dotsb
	\quad\text{and}\quad
	⨁_{n ≥ 1} ℝ = ⋃_{n ≥ 1} R_n \,.
\]

For every index~$k ≥ 1$ let~$i_k$ denote the restriction of~$ι_k$ to a map from~$ℝ^k$ to~$R_k$, let~$j_k$ denote the inclusion map from~$R_k$ to~$⨁_{n ≥ 1} ℝ$, and let~$f'_k$ denote the inclusion map from~$R_k$ to~$R_{k + 1}$.
We have hence the following commutative diagram for every~$k ≥ 1$:
\[
	\begin{tikzcd}[column sep = tiny]
		ℝ^k
		\arrow{rr}[above]{f_k}
		\arrow{d}[left]{i_k}
		\arrow[bend right, out = 255, in = 250]{ddr}[left]{ι_k}
		&
		{}
		&
		ℝ^{k + 1}
		\arrow{d}[right]{\,i_{k + 1}}
		\arrow[bend left, out = -255, in = -250]{ddl}[right]{\;ι_{k + 1}}
		\\
		R_k
		\arrow{rr}[above]{f'_k}
		\arrow{dr}[below left]{j_k}
		&
		{}
		&
		R_{k + 1}
		\arrow{dl}[below right]{j_{k + 1}}
		\\
		{}
		&
		⨁_{n ≥ 1} ℝ
		&
		{}
	\end{tikzcd}
\]
The maps~$i_k$ are bijective.
It follows that we have for every set~$X$ a chain of bijections
\begin{align*}
	\SwapAboveDisplaySkip
	{}&
	\{ \textstyle\text{maps $g \colon ⨁_{n ≥ 1} ℝ \to X$} \} \\
	≅{}&
	\left\{
		\begin{tabular}{l}
			sequences~$(g_n)_n$ of maps $g_n \colon R_n \to X$ \\
			with~$\restrict{g_{n + 1}}{\im ι_n} = g_n$ for all~$n ≥ 1$
		\end{tabular}
	\right\} \\
	={}&
	\left\{
		\begin{tabular}{l}
			sequences~$(g_n)_n$ of maps $g_n \colon R_n \to X$ \\
			with~$g_{n + 1} f'_n = g_n$ for all~$n ≥ 1$
		\end{tabular}
	\right\} \\
	≅{}&
	\left\{
		\begin{tabular}{l}
			sequences~$(h_n)_n$ of maps $h_n \colon ℝ^n \to X$ \\
			with~$h_{n + 1} i_{n + 1}^{-1} f'_n = h_n i_n^{-1}$ for all~$n ≥ 1$
		\end{tabular}
	\right\} \\
	={}&
	\left\{
		\begin{tabular}{l}
			sequences~$(h_n)_n$ of maps $h_n \colon ℝ^n \to X$ \\
			with~$h_{n + 1} f_n i_n^{-1} = h_n i_n^{-1}$ for all~$n ≥ 1$
		\end{tabular}
	\right\} \\
	={}&
	\left\{
		\begin{tabular}{l}
			sequences~$(h_n)_n$ of maps $h_n \colon ℝ^n \to X$ \\
			with~$h_{n + 1} f_n = h_n$ for all~$n ≥ 1$
		\end{tabular}
	\right\} \\
\end{align*}
given by
\[
	g
	\mapsto
	( \restrict{g}{R_n} )_n
	=
	( g j_n )_n
	\mapsto
	( g j_n i_n )_n
	=
	( g ι_n )_n \,.
\]
This shows that the set~$⨁_{n ≥ 1} ℝ$ together with the maps~$ι_n$ is a colimit of the diagram~\eqref{direct limit diagram}.



\subsubsection{On the Level of Vector Spaces}

Each of the maps~$f_n$ is linear, whence~\eqref{direct limit diagram} is a diagram in~$\Vect{ℝ}$.

Each of the maps~$ι_n$ is linear, whence~$⨁_{n ≥ 1} ℝ$ together with the maps~$ι_n$ is a cocone of the diagram~\eqref{direct limit diagram} in~$\Vect{ℝ}$.
To show that this cocone is a colimit, we need to show that we have for every vector space~$V$ a bijection
\begin{align*}
	{}&
	\{ \textstyle\text{linear maps~$g \colon ⨁_{n ≥ 1} ℝ \to V$} \} \\
	≅{}&
	\left\{
		\begin{tabular}{l}
			sequences~$(g_n)_n$ of linear maps $g_n \colon ℝ^n \to V$ \\
			with~$g_{n + 1} f_n = g_n$ for every~$n ≥ 1$
		\end{tabular}
	\right\}
\end{align*}
given by~$g \mapsto (g ι_n)_n$.
We already know that we have this bijection for maps between underlying sets, so it suffices to check that a map~$g \colon ⨁_{n ≥ 1} ℝ \to V$ is linear if and only if each composite~$g ι_n$ is linear.

If~$g$ is linear, then each composite~$g ι_n$ is again linear because~$ι_n$ is linear.

Suppose now that each composite~$g ι_n$ is linear.
Each~$R_n$ is a linear subspace of~$⨁_{n ≥ 1} R_n$, and each map~$i_n$ is an isomorphism of vector spaces.
So equivalently, each composite~$g ι_n i_n^{-1} = g j_n i_n i_n^{-1} = g j_n = \restrict{g}{R_n}$ is linear.
It follows from the following \lcnamecref{checking linearity on a filtration of subspaces} that~$g$ is therefore linear.

\begin{lemma}
	\label{checking linearity on a filtration of subspaces}
	Let~$V$ be a vector space and let~$(U_α)_{α ∈ A}$ be a family of subspaces of~$V$.
	Let~$W$ be another vector space and let~$f$ be a map from~$V$ to~$W$.
	Suppose that the following conditions hold:
	\begin{itemize}

		\item
			There exists for every two indices~$α$ and~$β$ another index~$γ$ with~$U_α, U_β ⊆ U_γ$.

		\item
			The vector space~$V$ is covered by the subspaces~$U_α$:
			we have~$V = ⋃_{α ∈ A} U_α$.

		\item
			The restriction~$\restrict{f}{U_α}$ is linear for every index~$α$.

	\end{itemize}
	The map~$f$ is then linear.
\end{lemma}

\begin{proof}
	Let~$x$ and~$y$ be two vectors in~$V$.
	There exists two indices~$α$ and~$β$ with~$x ∈ U_α$ and~$y ∈ U_β$.
	There further exists some index~$γ$ with~$U_α, U_β ⊆ U_γ$.
	We have~$x, y ∈ U_γ$, and hence also~$x + y ∈ U_γ$.
	We find that
	\[
		f (x + y)
		=
		\restrict{f}{U_γ} (x + y)
		=
		\restrict{f}{U_γ} x + \restrict{f}{U_γ} y
		=
		f x + f y \,.
	\]
	This shows that the map~$f$ is additive.

	There exists for every scalar~$λ$ and every vector~$v$ in~$V$ some index~$α$ such that~$x ∈ U_α$, and therefore also~$λ x ∈ U_α$.
	We find that
	\[
		f λ x
		=
		\restrict{f}{U_α} λ x
		=
		λ \restrict{f}{U_α} x
		=
		λ f x \,.
	\]
	This shows that the map~$f$ is homogeneous.
\end{proof}



\subsubsection{On the Level of Topological Spaces}

Each of the maps~$f_n$ is continuous, whence~\eqref{direct limit diagram} is a diagram~$\Top$.

We want to endow the direct sum~$⨁_{n ≥ 1} ℝ$ with a topology such that~$⨁_{n ≥ 1} ℝ$ together with the maps~$ι_n$ is a colimit of the diagram~\eqref{direct limit diagram} in~$\Top$.
This topology on~$⨁_{n ≥ 1} ℝ$ must therefore satisfy the following two conditions:
\begin{itemize*}

	\item
		The resulting topological space~$⨁_{n ≥ 1} ℝ$ together with the maps~$ι_n$ needs to be a cocone of the diagram~\eqref{direct limit diagram} in~$\Top$.

		We already know that~$ι_{n + 1} f_n = ι_n$ for every~$n ≥ 1$, so this condition is equivalent to each map~$ι_n$ being continuous.

	\item
		We need to have for every topological space~$X$ a bijection
		\begin{align*}
			{}&
			\{ \textstyle\text{continuous maps~$g \colon ⨁_{n ≥ 1} ℝ \to X$} \} \\
			≅{}&
			\left\{
				\begin{tabular}{l}
					sequences~$(g_n)_n$ of continuous maps $g_n \colon ℝ^n \to X$ \\
					with~$g_{n + 1} f_n = g_n$ for every~$n ≥ 1$
				\end{tabular}
			\right\}
		\end{align*}
		given by~$g \mapsto (g ι_n)_n$.

		We already know that we have this bijection for maps between underlying sets.
		This condition is therefore equivalent to~$g \colon ⨁_{n ≥ 1} ℝ \to X$ being continuous if and only if each composite~$g ι_n$ is continuous.

\end{itemize*}

For every~$n ≥ 1$ let~$\top{T}_n$ be the topology on~$ℝ^n$.
The induced topology
\[
	\top{T} ≔ ⋀_{n ≥ 1} {} (ι_n)_* \top{T}_n
\]
(in the sense of \cref{universal properties for pullback and pushforward of topologies} and \cref{notation for infima and suprema of topologies}) is the finest topology on the set~$⨁_{n ≥ 1} ℝ$ with respect to which each~$ι_n$ is continuous.
For every topological space~$(X, \top{T}_X)$ and every map~$g \colon ⨁_{n ≥ 1} ℝ \to X$ we have the chain of equivalences
\begin{align*}
	{}&
	\text{$g$ is continuous with respect to~$\top{T}$ and~$\top{T}_X$} \\
	\iff{}&
	g^* \top{T}_X ⊆ \top{T} \\
	\iff{}&
	\textstyle g^* \top{T}_X ⊆ ⋀_{n ≥ 1} {} (ι_n)_* \top{T}_n \\
	\iff{}&
	\text{$g^* \top{T}_X ⊆ (ι_n)_* \top{T}_n$ for every~$n ≥ 1$} \\
	\iff{}&
	\text{$\top{T}_X ⊆ g_* (ι_n)_* \top{T}_n$ for every~$n ≥ 1$} \\
	\iff{}&
	\text{$\top{T}_X ⊆ (g ι_n)_* \top{T}_n$ for every~$n ≥ 1$} \\
	\iff{}&
	\text{$g ι_n$ is continuous with respect to~$\top{T}_n$ and~$\top{T}_X$ for every~$n ≥ 1$} \,.
\end{align*}
We have thus found the desired topology on~$⨁_{n ≥ 1} ℝ$:
it is given by~$⋀_{n ≥ 1} {} (ι_n)_* \top{T}_n$, which is also the finest topology making all~$ι_n$ continuous.
We have more explicitly for every subset~$U$ of~$⨁_{n ≥ 1} ℝ$ the chain of equivalences
\begin{align*}
	{}&
	U ∈ \top{T} \\
	\iff{}&
	\textstyle\text{$U ∈ ⋀_{n ≥ 1} {} (ι_n)_* \top{T}_n$} \\
	\iff{}&
	\text{$U ∈ (ι_n)_* \top{T}_n$ for every~$n ≥ 1$} \\
	\iff{}&
	\text{$ι_n^{-1} U ∈ \top{T}_n$ for every~$n ≥ 1$} \\
	\iff{}&
	\text{$ι_n^{-1} U$ is open in~$ℝ^n$ for every~$n ≥ 1$} \\
	\iff{}&
	\text{$U ∩ ℝ^n$ is open in~$ℝ^n$ for every~$n ≥ 1$} \,.
\end{align*}
