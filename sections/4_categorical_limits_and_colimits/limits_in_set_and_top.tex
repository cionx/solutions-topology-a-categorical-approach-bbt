\subsection{Limits in~\texorpdfstring{$\Set$}{Set} and~\texorpdfstring{$\Top$}{Top}}



\subsubsection{In~\texorpdfstring{$\Set$}{Set}}

Let~$F \colon \cat{D} \to \Set$ be a diagram in~$\Set$.

For every object~$Y$ of~$\cat{D}$ let~$π_Y$ be the canonical projection from the product~$∏_{X ∈ \cat{D}} F X$ onto its factor~$F Y$.
Let~$E$ be the subset of the product~$∏_{X ∈ \cat{D}} F X$ given by
\begin{align*}
	{}&
	E \\
	≔{}&
	\{
		\textstyle (a_X)_X ∈ \prod_{X ∈ \cat{D}} F X
		\suchthat
		\text{$(F f) a_X = a_Y$ for every morphism~$f \colon X \to Y$ in~$\cat{D}$}
	\} \\
	={}&
	\{
		\textstyle a ∈ ∏_{X ∈ \cat{D}} F X
		\suchthat
		\text{$(F f) π_X a = π_Y a$ for every morphism~$f \colon X \to Y$ in~$\cat{D}$}
	\} \,.
\end{align*}
Let~$ι$ be the inclusion map from~$E$ into the product~$∏_{X ∈ \cat{D}} F X$, and for every object~$X$ of the indexing category~$\cat{D}$ let~$p_X$ be the composite~$π_X ι$.

We have for every morphism~$f \colon X \to Y$ in~$\cat{D}$ the equalities
\[
	(F f) p_X
	=
	(F f) π_X ι
	=
	π_Y ι
	=
	p_Y \,.
\]
This tells us that the set~$E$ together with the maps~$p_X$, where~$X$ ranges over the objects of~$\cat{D}$, is a cone on~$F$.
We have more generally for every set~$S$ the chain of bijections
\begin{align*}
	{}&
	\{ \text{maps~$\textstyle h \colon S \to E$} \} \\
	≅{}&
	\{ \text{maps~$\textstyle k \colon S \to ∏_{X ∈ \cat{D}} F X$ with~$k S ⊆ E$} \} \\
	={}&
	\left\{
		\begin{tabular}{l}
			maps~$k \colon S \to ∏_{X ∈ \cat{D}} F X$ with~$(F f) π_X k = π_Y k$ \\
			for every morphism~$f \colon X \to Y$ in~$\cat{D}$
		\end{tabular}
	\right\} \\
	≅{}&
	\left\{
		\begin{tabular}{l}
			families~$(k_X)_{X ∈ \cat{D}}$ of maps~$k_X \colon S \to F X$ with~$(F f) k_X = k_Y$ \\
			for every morphism~$f \colon X \to Y$ in~$\cat{D}$
		\end{tabular}
	\right\}
\end{align*}
given by
\[
	h \mapsto ι h \mapsto (π_X ι h)_{X ∈ \cat{D}} = (p_X h)_{X ∈ \cat{D}} \,.
\]
This tells us that the set~$E$ together with the maps~$p_X$, where~$X$ ranges over the objects of~$\cat{D}$, is a limit of the diagram/functor~$F$.



\subsubsection{In~$\Top$}

Suppose now that the codomain of the functor~$F$ is~$\Top$ instead of~$\Set$.

We endow the product~$∏_{X ∈ \cat{D}} F X$ with the product topology, which makes the projection maps~$π_X$ continuous.
We further endow~$E$ with the subspace topology induced from the product topology, which makes the inclusion map~$ι$ continuous.
It follows that the maps~$p_X = π_X ι$ are continuous.

Let~$S$ be another topological space.
A map~$h \colon S \to E$ is continuous if and only if the composite~$h π$ is continuous, and a map~$k \colon S \to ∏_{X ∈ \cat{D}} F X$ is continuous if and only if each composite~$π_X k$ is continuous.
We hence find that the bijection
\begin{align*}
	{}&
	\{ \text{maps~$\textstyle h \colon S \to E$} \} \\
	≅{}&
	\left\{
		\begin{tabular}{l}
			families~$(k_X)_{X ∈ \cat{D}}$ of maps~$k_X \colon S \to F X$ with~$(F f) k_X = k_Y$ \\
			for every morphism~$f \colon X \to Y$ in~$\cat{D}$
		\end{tabular}
	\right\}
\end{align*}
given by~$h \mapsto (p_X h)_{X ∈ \cat{D}}$ restricts to a bijection
\begin{align*}
	{}&
	\{ \text{continuous maps~$\textstyle h \colon S \to E$} \} \\
	≅{}&
	\left\{
		\begin{tabular}{l}
			families~$(k_X)_{X ∈ \cat{D}}$ of continuous maps~$k_X \colon S \to F X$ \\
			with~$(F f) k_X = k_Y$ for every morphism~$f \colon X \to Y$ in~$\cat{D}$
		\end{tabular}
	\right\} \,.
\end{align*}
This tells us that the topological space~$E$ together with the projection maps~$p_X$, where~$X$ ranges over the objects of~$\cat{D}$, is a limit of the diagram~$F$ in~$\Top$.
