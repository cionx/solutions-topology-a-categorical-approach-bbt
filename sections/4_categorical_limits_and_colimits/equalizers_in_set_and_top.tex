\subsection{Construction of Equalizers in~\texorpdfstring{$\Set$}{Set} and~\texorpdfstring{$\Top$}{Top}}



\subsubsection{In~\texorpdfstring{$\Set$}{Set}}

We consider the diagram
\begin{equation}
	\label{equalizer diagram}
	\begin{tikzcd}
		X
		\arrow[shift left]{r}[above]{f}
		\arrow[shift right]{r}[below]{g}
		&
		Y
	\end{tikzcd}
\end{equation}
in~$\Set$, the subset
\[
	E ≔ \{ x ∈ X \suchthat f x = g x \}
\]
of~$X$ and the inclusion map~$i$ from~$E$ to~$X$.
We have~$f i = g i$, and we have for every set~$Z$ and every map~$h \colon X \to Y$ the chain of equivalences
\begin{align*}
	{}&
	f h = g h
	\iff
	\text{$h Z ⊆ E$}
	\iff
	\textstyle\text{there exists a map~$h' \colon Z \to E$ with~$h = i h'$} \,,
\end{align*}
see~\cref{universal property of equalizer as diagram}.
\begin{figure}
	\[
		\begin{tikzcd}
			{}
			&
			Z
			\arrow{d}[right]{h}
			\arrow[dashed]{dl}[above left]{h'}
			&
			{}
			\\
			E
			\arrow{r}[above, pos = 0.6]{i}
			&
			X
			\arrow[shift left]{r}[above]{f}
			\arrow[shift right]{r}[below]{g}
			&
			Y
		\end{tikzcd}
	\]
	\caption{Universal property of the equalizer.}
	\label{universal property of equalizer as diagram}
\end{figure}
The map~$h'$ in question is then unique because the map~$i$ is injective and thus a monomorphism in~$\Set$.

This shows that the set~$E$ together with the map~$i$ is a limit of the diagram~\eqref{equalizer diagram} in~$\Set$.



\subsubsection{In~\texorpdfstring{$\Top$}{Top}}

Suppose now that~\eqref{equalizer diagram} is a diagram in~$\Top$.

We can then endow~$E$ with the subspace topology induced from the topology of~$X$.
This ensures that the inclusion map~$i$ is continuous, and that more generally a map~$h'$ from~$X$ to~$E$ is continuous if and only if its composite~$i h'$ is continuous.
The bijection
\[
	\{ \text{maps~$\textstyle h' \colon Z \to E$} \}
	≅
	\{ \text{maps~$\textstyle h \colon Z \to X$ with~$f h = g h$} \} \,,
	\quad
	h' \mapsto i h'
\]
therefore restricts to a bijection
\begin{align*}
	{}&
	\{ \textstyle \text{continuous maps~$h' \colon Z \to E$} \} \\
	≅{}&
	\{ \textstyle \text{continuous maps~$h \colon Z \to X$ with~$f h = g h$} \} \,.
\end{align*}
This shows that the topological space~$E$ together with the map~$i$ is a limit of the diagram~\eqref{equalizer diagram} in~$\Top$.
