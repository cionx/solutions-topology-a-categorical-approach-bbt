\subsection{Theorem~3.5}

The presented proof of Theorem~3.5 can be streamlined thanks to the following observation(s).

\begin{proposition}
	Let~$X$ be a topological space and let~$x$ be a point in~$X$ admitting a countable neighbourhood basis~$\{ U_n \suchthat n ≥ 0 \}$.
	For every~$n ≥ 0$ let~$V_n = U_1 ∩ \dotsb ∩ U_n$.
	Then~$\{ V_n \suchthat n ≥ 0 \}$ is again a neighbourhood basis of~$x$.
\end{proposition}

\begin{proof}
	The sets~$V_n$ are again open neighbourhoods of~$x$, since each~$U_n$ is an open neighbourhood of~$x$.
	There exists for every neighbourhood~$U$ of~$x$ some index~$n$ with~$U_n ⊆ U$.
	We have~$V_n ⊆ U_n$, and therefore~$V_n ⊆ U$.
\end{proof}

\begin{corollary}
	\label{first countable spaces have linearly ordered neighbourhood bases}
	Let~$X$ be a first countable topological space.
	Every point~$x$ in~$X$ admits a countable neighbourhood basis~$\{ U_n \suchthat n ≥ 0 \}$ such that~$U_{n + 1} ⊆ U_n$ for every~$n ≥ 0$.
	\qed
\end{corollary}

In the presented proof of Theorem~3.5 we may additionally assume that
\[
	U_0 ⊇ U_1 ⊇ U_2 ⊇ U_3 ⊇ \dotsb
	\quad\text{and}\quad
	V_0 ⊇ V_1 ⊇ V_2 ⊇ V_3 ⊇ \dotsb \,.
\]
This then implies that the sequence~$(x_n)_n$ itself converges to both~$x$ and~$y$.
There is hence no need for subsequences.
