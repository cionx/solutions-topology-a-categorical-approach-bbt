\subsection{Exercise~3.13}

The book confuses the roles of the lattices:
in the lattice on the left there are maximal filters that are not prime, but every prime filter is also maximal.
In the lattice on the right there are prime filters that are not maximal, but every maximal filter is also prime.

We will use the following observation to simplify our argumentation.

\begin{definition}
	Let~$P$ be a poset.
	A filter~$\filter{F}$ in~$P$ is \defemph{principal} if there exists an element~$x$ of~$P$ with~$\filter{F} = \{ y ∈ P \suchthat y ≥ x \}$.
\end{definition}

\begin{proposition}
	Let~$P$ be a poset admitting binary meets (and thus finite meets by induction).
	Every finite filter in~$P$ if principal.
\end{proposition}

\begin{proof}
	Let~$\filter{F}$ be a filter in~$P$ consisting of finitely many elements~$x_1, \dotsc, x_n$.
	The meet~$x ≔ x_1 ∧ \dotsb ∧ x_n$ exists in~$P$, and it is again contained in~$\filter{F}$.
	It follows that~$\filter{F} = \{ y ∈ P \suchthat y ≥ x \}$ because~$\filter{F}$ is upward closed.
\end{proof}

\begin{corollary}
	In a finite lattice, every filter is principal.%
	\footnote{
		By a \defemph{lattice} we mean poset in which every two elements admit both a meet and a join, and which admits both a greatest element and a least element.
	}
	\qed
\end{corollary}


\subsubsection{Prime but not maximal}

The filters in the first lattice are
\[
	\{ ⊤ \} \,, \quad
	\{ x, ⊤ \} \,, \quad
	\{ y, ⊤ \} \,, \quad
	\{ z, ⊤ \} \,, \quad
	\{ ⊥, x, y, z \} \,.
\]
The last filter is improper and therefore neither maximal nor prime.
The filters~$\{ x, ⊤ \}$,~$\{ y, ⊤ \}$ and~$\{ z, ⊤ \}$ are maximal, but they are not prime;
we have, for example,~$y ∨ z = ⊤ ∈ \{ x, ⊤ \}$ but neither~$y$ nor~$z$ is an element of~$\{ x, ⊤ \}$.
The filter~$\{ ⊤ \}$ is neither maximal nor prime.




\subsubsection{Maximal but not prime}

The Hasse diagram of the second lattice is somewhat misleading, it should be drawn as follows instead:
\[
	\begin{tikzcd}[row sep = normal]
		⊤
		\\
		v
		\arrow[dash]{u}
		\\
		u
		\arrow[dash]{u}
		\\
		\arrow[dash]{u}
		⊥
	\end{tikzcd}
\]
This makes it clear that we are dealing with a linear poset.

\begin{proposition}
	In a linear poset, every proper filter is prime.
\end{proposition}

\begin{proof}
	Let~$\filter{F}$ be a filter in a linear poset~$L$.
	Let~$x$ and~$y$ be two elements of~$L$ such that~$x ∨ y$ belongs to~$\filter{F}$.
	We have~$x ∨ y = \max(x, y) ∈ \{ x, y \}$ because~$L$ is linear, and therefore either~$x ∈ \filter{F}$ or~$y ∈ \filter{F}$.
\end{proof}

The filters on the second lattice are
\[
	\{ ⊤ \} \,, \quad
	\{ v, ⊤ \} \,, \quad
	\{ u, v, ⊤ \} \,, \quad
	\{ ⊥, u, v, ⊤ \} \,.
\]
The last filter is improper and therefore neither maximal nor prime.
The filter~$\{ u, v, ⊤ \}$ is maximal and also prime.
The filters~$\{ v, ⊤ \}$ and~$\{ ⊤ \}$ are prime, but not maximal.



% TODO: Proof of the note?
