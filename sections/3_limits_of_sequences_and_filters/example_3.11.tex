\subsection{Example~3.11}

The sequence~$(x_n)_n$ is always contained in~$X$, whence~$X$ belongs to~$\event_{(x_n)_n}$.

Let~$A$ be a set belonging to~$\event_{(x_n)_n}$ and let~$B$ be a subset of~$X$ with~$A ⊆ B$.
There exists some index~$N$ with~$x_n ∈ A$ for every~$n ≥ N$.
It then follows that also~$x_n ∈ B$ for every~$n ≥ N$.
Therefore,~$B$ again belongs to~$\event_{(x_n)_n}$.

Let~$A$ and~$A'$ be two sets belonging to~$\event_{(x_n)_n}$.
There exist indices~$N$ and~$N'$ with~$x_n ∈ A$ for every~$n ≥ N$ and similarly~$x_n ∈ A'$ for every~$n ≥ N'$.
It follows for~$N'' ≔ \max(N, N')$ that~$x_n ∈ A ∩ A'$ for every~$n ≥ N''$.
This shows that the intersection~$A ∩ A'$ again belongs to~$\event_{(x_n)_n}$.
