\subsection{The Ultrafilter Lemma}

The required upper bound for a chain of filters can also be constructed as explained in the following \lcnamecref{union of chain of filters is again a filter}.

\begin{proposition}
	\label{union of chain of filters is again a filter}
	Let~$(\filter{F}_α)_{α ∈ A}$ be a nonempty chain of filters in a poset~$P$.
	\begin{enumerate}

		\item
			The union~$⋃_{α ∈ A} \filter{F}_α$ is again a filter in~$P$.

		\item
			If~$P$ admits a least element and each~$\filter{F}_α$ is proper, then~$⋃_{α ∈ A} \filter{F}_α$ is again proper.

	\end{enumerate}
\end{proposition}

\begin{proof}
	Let~$\filter{F} ≔ ⋃_{α ∈ A} \filter{F}_α$.

	There exists an index~$α$ in~$A$, and the filter~$\filter{F}_α$ is nonempty.
	Therefore,~$\filter{F}$ is nonempty.

	Let~$b$ belong to~$\filter{F}$, and let~$c$ be some element of~$P$ with~$b ≤ c$.
	The element~$b$ belongs to~$\filter{F}_α$ for some index~$α$.
	It follows that~$c$ also belongs to~$\filter{F}_α$, because~$\filter{F}_α$ is upward closed.
	Consequently,~$c$ belongs to~$\filter{F}$.

	Let~$b$ and~$c$ be two elements belonging to~$\filter{F}$.
	This means that there exist indices~$β$ and~$γ$ such that~$b$ belongs to~$\filter{F}_β$ and~$c$ belongs to~$\filter{F}_γ$.
	We have by assumption either~$\filter{F}_β ⊆ \filter{F}_γ$ or~$\filter{F}_γ ⊆ \filter{F}_β$.
	It follows that for some index~$α ∈ \{ β, γ \}$ that both~$b$ and~$c$ belong to~$\filter{F}_α$.
	It then further follows that there exists some element~$a$ belonging to~$\filter{F}_α$ with~$a ≤ b, c$.
	The element~$a$ again belongs to~$\filter{F}$.

	Suppose now that each~$\filter{F}_α$ is proper, and that~$P$ admits a least element~$0$.
	That the~$\filter{F}_α$ are proper means that no~$\filter{F}_α$ contains~$0$.
	Consequently,~$\filter{F}$ again doesn’t contain~$0$, so that~$\filter{F}$ is proper.
\end{proof}
