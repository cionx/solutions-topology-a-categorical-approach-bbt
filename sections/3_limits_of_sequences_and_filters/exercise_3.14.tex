\subsection{Exercise~3.14}

The given formula
\[
	\text{\enquote{$ α ≔ \{ \text{ordinals~$β < α$} \} $}}
\]
doesn’t make sense since it defines~$α$ in terms of itself, and since no order has been defined so far on the class of ordinals.
This formula also seems requires some form of transfinite recursion to already be available.

For our purposes, we call a well-ordered set~$S$ is an ordinal if (and only if) it satisfies~$x = \{ y ∈ S \suchthat y < x \}$ for every element~$x$ of~$S$.
Ordinals are thus well-ordered sets that have a very specific set-theoretic form.
As examples, the only one-element ordinal is~$\{ ∅ \}$, the only two-element ordinal is~$\{ ∅, \{ ∅ \} \}$, and the only three-element ordinal is~$\{ ∅, \{ ∅ \}, \{ ∅, \{ ∅ \} \} \}$.
Any two isomorphic ordinals must already be set-theoretically equal (both in terms of their elements and their order relation), and every well-ordered set is isomorphic to precisely one ordinal.%
\footnote{
	This is not at all obvious, and needs some transfinite induction/recursion to establish.
}
It follows from this definition that every ordinal is the set of all its predecessors, i.e.,~$α = \{ β \suchthat \text{$β$ is an ordinal with~$β < α$} \}$ for every ordinal~$α$.



\subsubsection{(i)}

Suppose that there exists a cardinal~$κ$ that is strictly larger than any ordinal.
We can consider for every set~$X$ its power set, and for every subset~$A$ of~$X$ we can consider the set of all well-orders on~$A$.
This allows us to consider the set~$W$ of all well-ordered sets whose underlying set is a subset of~$κ$.
Let~$∼$ be the equivalence relation on~$W$ for which two elements~$A$ and~$B$ of~$W$ are equivalent if and only if they are isomorphic.
By choice of~$κ$, the elements of~$W / {∼}$ are in bijection with the ordinals.
But this would imply that the ordinals form a set, which contradicts the Burali-Forti paradox.



\subsubsection{(ii)}

Let~$α$ be an arbitrary ordinal.
The collection
\[
	S_α ≔ \{ β \suchthat \text{$β$ is an ordinal with~$β ≤ α$} \}
\]
is a set (namely~$α ∪ \{ α \}$, i.e., the successor of~$α$).
Suppose that for some element~$β$ of~$S_α$ the property~$P β$ were to be false, i.e., suppose that the set
\[
	F ≔ \{ β ∈ S_α \suchthat \text{$P β$ is false} \}
\]
were nonempty.
The set~$S_α$ is well-ordered, so there would exist some minimal element~$β$ of~$F$.
But by assumption~$β$ can be neither~$0$, nor a successor ordinal, nor a limit ordinal.
So~$β$ cannot exist!

This shows that the statement~$P β$ is true for every ordinal~$α$ and every ordinal~$β$ with~$β ≤ α$.
Consequently,~$P α$ is true for every ordinal~$α$.




\subsubsection{(iii)}

The given function~$f$ satisfies~$f a > a$ for every element~$a$ of~$\mathcal{P}$.

The author has absolutely no idea how one is supposed to solve this part of the exercise with only transfinite induction.
Instead, we will use transfinite recursion, which is proven in~\autocite[§16, p.~53]{halmos_naive_set_theory} using transfinite induction.

The set~$\mathcal{P}$ is nonempty, so there exists by assumption an element~$x$ of~$\mathcal{P}$.
We define in the following for every ordinal~$α$ an element~$f^α x$ of~$\mathcal{P}$ such that~$f^α x < f^β x$ whenever~$α < β$.

We start (as usual) with~$f^0 x ≔ x$.
In terms of successor ordinals, we define~$f^{α + 1} x ≔ f f^α x$ for every ordinal~$α$.
For every ordinal~$β < α + 1$ we then have~$β ≤ α$ and thus
\[
	f^β x ≤ f^α x < f f^α x = f^{α + 1} x \,.
\]
Given now a limit ordinal~$λ$, we have already defined~$f^α$ for every ordinal~$α$ with~$α < λ$, and the set~$C_α ≔ \{ f^α \suchthat α < λ \}$ is a chain in~$\mathcal{P}$.
There then exists an upper bound~$u_λ$ for~$C_α$ in~$\mathcal{P}$, and we define~$f^λ ≔ f u_λ$.
For every ordinal~$α$ with~$α < λ$ we have
\[
	f^α x ≤ u_λ < f u_λ \,.
\]

For every ordinal~$α$ we have a chain of length~$α$ in~$\mathcal{P}$, given by the function
\[
	α \to \mathcal{P} \,, \quad β \mapsto f^β x \,.
\]
This function is in particular injective, whence we have~$\card{α} ≤ \card{\mathcal{P}}$ for every ordinal~$α$.
But we know from part~(i) of this exercise that there exists for every cardinal~$κ$ some ordinal~$α$ with~$κ ≤ \card{α}$.
It follows that~$κ ≤ \card{\mathcal{P}}$ for every cardinal~$κ$, which is wrong.

This contradiction shows that the function~$f$ cannot exist.
Hence, Zorn’s lemma must hold.
