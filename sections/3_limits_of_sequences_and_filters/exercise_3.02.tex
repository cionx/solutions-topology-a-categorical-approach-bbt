\subsection{Exercise~3.2}
\label{exercise 3.2}

\begin{lemma}
	\label{uniquess of sequential limits is inherited by finer topologies}
	Let~$X$ be a set and let~$\top{T}_1$ and~$\top{T}_2$ be two topologies on~$X$.
	Suppose that limits of sequences with respect to~$\top{T}_1$ are unique and that the topology~$\top{T}_2$ is finer than the topology~$\top{T}_1$.
\end{lemma}

\begin{proof}
	Let~$(x_n)_n$ be a sequence in~$X$ with limits~$x$ and~$y$ with respect to~$\top{T}_2$.
	Then~$x$ and~$y$ are also limits of~$(x_n)_n$ with respect to~$\top{T}_1$, since~$\top{T}_1$ is coarser than~$\top{T}_2$.
	Consequently,~$x = y$.
\end{proof}

Let~$X$ be the real line together with the cocountable topology.
Every two proper open subsets of~$X$ intersect (because the union of two countable subsets is again a proper subset of~$X$).
Therefore, no two points of~$X$ can be separated by disjoint open neighbourhoods:
the space~$X$ is very much not a Hausdorff space.

Let~$(x_n)_n$ be a sequence in~$X$ that converges to a point~$x$.
We note that the set~$U ≔ X ∖ \{ x_n \suchthat n ≥ 0 \}$ is open in~$X$, and that the sequence~$(x_n)_n$ never enters the set~$U$.
Therefore,~$x$ cannot lie in~$U$.
More generally, every subsequence~$(x_{k_i})_i$ also converges to~$x$, whence~$x$ cannot be contained in~$X ∖ \{ x_{k_i} \suchthat i ≥ 0 \}$.

The point~$x$ must thus be contained in every subsequence of~$(x_n)_n$.
This is only possible if the sequence~$(x_n)_n$ is eventually constant with value~$x$, i.e., if there exists some index~$N$ with~$x_n = x$ for every~$n ≥ N$.

Consequently, the possible limits of the sequence~$(x_n)_n$ are the same as the limits of the constant sequence~$x, x, \dotsc$
The cocountable topology is finer than the cofinite topology, so it follows from Theorem~3.1 (and the above \cref{uniquess of sequential limits is inherited by finer topologies}) that~$x$ is the unique limit of~$(x_n)_n$.
