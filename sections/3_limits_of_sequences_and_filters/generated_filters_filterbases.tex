\subsection{Generated Filters and Filterbases}

\begin{proposition}
	Let~$X$ be a set and let~$(\filter{F}_α)_{α ∈ A}$ be a family of filters on~$X$.
	The intersection~$⋂_{α ∈ A} \filter{F}_α$ is again a filter on~$X$.
\end{proposition}

\begin{proof}
	We denote the intersection~$⋂_{α ∈ A} \filter{F}_α$ by~$\filter{F}$.

	Let~$A$ and~$B$ be two sets belonging to~$\filter{F}$.
	This means that both~$A$ and~$B$ are contained in each~$\filter{F}_α$.
	Consequently, the intersection~$A ∩ B$ is again contained in each~$\filter{F}_α$.
	Therefore,~$A ∩ B$ is contained in~$\filter{F}$.

	Each~$\filter{F}_α$ is nonempty and upwards closed, and thus contains the set~$X$.
	Therefore,~$X$ is also contained in~$\filter{F}$.
	This shows that~$\filter{F}$ is nonempty.

	Let~$A$ be any set belonging to~$\filter{F}$ and let~$B$ be any subset of~$X$ with~$A ⊆ B$.
	The set~$A$ belongs to each~$\filter{F}_α$, so~$B$ belongs to each~$\filter{F}_α$.
	Hence,~$B$ belongs to~$\filter{F}$.
\end{proof}

\begin{corollary}
	Let~$X$ be a set and let~$\filterbase{S}$ be a collection of subsets of~$X$, i.e., a subset of the power set~$2^X$.
	There exists a smallest filter on~$X$ containing~$\filterbase{S}$.
	\qed
\end{corollary}

\begin{definition}
	Let~$X$ be a set and let~$\filterbase{S}$ be a collection of subsets of~$X$.
	The smallest filter on~$X$ containing~$\filterbase{S}$ is the \defemph{filter generated by~$\filterbase{S}$}.
\end{definition}

\begin{proposition}
	Let~$X$ be a set and let~$\filterbase{B}$ be a collection of subsets of~$X$.
	Suppose that~$\filterbase{B}$ is nonempty and downwards directed.
	Then the filter~$\filter{F}$ generated by~$\filterbase{B}$ is given by
	\[
		\{
			A ⊆ X
			\suchthat
			\text{there exists~$B ∈ \filterbase{B}$ with~$B ⊆ A$}
		\} \,.
	\]
\end{proposition}

\begin{proof}
	We denote the proposed set by~$\filter{G}$.
	The filter~$\filter{F}$ contains~$\filterbase{B}$ and is upwards closed, and therefore also contains~$\filter{G}$, i.e.,~$\filter{G} ⊆ \filter{F}$.
	To prove the inclusion~$\filter{F} ⊆ \filter{G}$ we will show that~$\filter{G}$ is a filter.

	Let~$A$ and~$A'$ be two sets belonging to~$\filter{G}$.
	This means that there exist two sets~$B$ and~$B'$ belonging to~$\filterbase{B}$ with~$B ⊆ A$ and~$B' ⊆ A'$.
	There exists by assumption a set~$B''$ belonging to~$\filterbase{B}$ with~$B'' ⊆ B ∩ B'$, and thus~$B'' ⊆ A ∩ A'$.
	The intersection~$A ∩ A'$ therefore again belongs to~$\filter{G}$.

	There exists some set belonging to~$\filterbase{B}$, whence the set~$X$ belongs to~$\filter{G}$.
	Therefore,~$\filter{G}$ is nonempty.

	Let~$A$ be a set belonging to~$\filter{G}$ and let~$A'$ be a subset of~$X$ with~$A ⊆ A'$.
	There exist some set~$B$ belonging to~$\filterbase{B}$ with~$B ⊆ A$, and thus~$B ⊆ A'$.
	Therefore,~$A'$ again belongs to~$\filter{G}$.
\end{proof}

\begin{proposition}
	Let~$X$ be a set and let~$\filterbase{S}$ be a collection of subsets of~$X$.
	Let~$\filter{F}$ be the filter generated by~$\filterbase{S}$.
	A filterbase for~$\filter{F}$ is given by the collection of finite intersections of sets belonging to~$\filterbase{S}$.
\end{proposition}

\begin{proof}
	Let
	\[
		\filterbase{B}
		≔
		\{
			A_1 ∩ \dotsb ∩ A_n
			\suchthat
			n ≥ 0,
			A_1, \dotsc, A_n ∈ \filterbase{S}
		\}
	\]
	and let~$\filter{F}'$ be the filter generated by~$\filterbase{B}$.
	We have~$\filterbase{S} ⊆ \filterbase{B} ⊆ \filter{F}'$ and thus~$\filter{F} ⊆ \filter{F}'$, because~$\filter{F}'$ is a filter.
	It follows from the inclusion~$\filterbase{S} ⊆ \filter{F}$ that also~$\filterbase{B} ⊆ \filter{F}$, since~$\filter{F}$ is a filter, and therefore~$\filter{F}' ⊆ \filter{F}$, again because~$\filter{F}$ is a filter.
\end{proof}
