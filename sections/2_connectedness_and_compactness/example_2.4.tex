\subsection{Example~2.4}



\subsubsection{$X$ is path connected}

The horizontal line~$L ≔ [0, 1] × \{ 0 \}$ is homeomorphic to the interval~$[0, 1]$ and therefore path-connected.
For every~$x ∈ C$, the tooth~$T_x ≔ \{ x \} × [0, 1]$ of the comb~$X$ is also homeomorphic to~$[0, 1]$ and therefore path-connected.
The line~$L$ and tooth~$T_x$ intersect in the point~$(x, 0)$ of~$X$, and are therefore both contained in the same path component of~$X$.

It follows that the path component of~$X$ that contains~$L$ also contains every tooth~$T_x$ with~$x ∈ C$, and contains therefore all of~$X$.
In other words, the space~$X$ is path connected.



\subsubsection{$X$ is not locally path connected}

We start by making an important observation.

Let~$π \colon X \to [0, 1]$ be th projection onto the first coordinate
Let~$x$ and~$x'$ be two points in~$C$ and let~$γ \colon [0, 1] \to X$ be a path from~$y ≔ (x, 1)$ to~$y' ≔ (x', 1)$.
There exists a number~$x''$ between~$x$ and~$x'$ that is not contained in~$C$.
By the intermediate value theorem, there exists some~$t ∈ [0, 1]$ with~$π γ t = x''$.
The original point~$γ t$ in~$X$ needs to be given by~$γ t = (x'', 0)$ since this is the only preimage of~$x''$ under~$π$.

This tells us that any path between two tips of the comb needs to pass through the horizontal line~$L$ at some point.
See also \cref{path must pass through the horizontal line}.
\begin{figure}
	\centering
	\begin{tikzpicture}
		% horizontal line
		\draw (-2, 0) node[left] {$\cdots$} -- (5, 0) node[right] {$\cdots$};
		% vertical lines
		\draw (-1.5, 0) -- (-1.5, 3) ;
		\draw (0, 0) -- (0, 3) node[above] {$(x, 1)$};
		\draw (2, 0) -- (2, 3) node[above] {$(x', 1)$};
		\draw (4.5, 0) -- (4.5, 3);
		% the path
		\draw[very thick, ->] (0, 3) -- (0, 0) -- (2, 0) -- (2, 3);
		% point between the teeths of the comb
		\draw (1, 0.1) -- (1, -0.1) node[below] {$(x'', 0)$};
	\end{tikzpicture}
	\caption{Every paths between tips must pass through the horizontal line.}
	\label{path must pass through the horizontal line}
\end{figure}

Suppose that the space~$X$ were locally path connected.
We consider then the point~$p ≔ (0, 1)$ in the upper-left corner of the comb, and its open neighbourhood
\[
	V ≔ [0, 1] × (1/2, 1]
\]
that consists of the open upper half of the comb.
There exists by assumption a path-connected neighbourhood~$U$ of~$p$ with~$U ⊆ V$.
There exists a radius~$ε > 0$ with~$\ball(p, ε) ∩ X ⊆ U$.
For a sufficiently large integer~$n$ we have~$1/n < ε$.
The two comb tips~$y ≔ (1/(n + 1), 1)$ and~$y' ≔ (1/n, 1)$ are contained in~$\ball(p, ε) ∩ X$, and are therefore contained in~$U$.
There exists a path~$γ$ from~$y$ to~$y'$ in~$U$ because~$U$ is path connected.

We can regard~$γ$ as a path from~$y$ to~$y'$ in~$X$.
We have seen above that the path~$γ$ needs to pass through the horizontal line~$L$ at some point.
But this cannot be, because~$γ$ is completely contained in~$V$ and~$V$ is disjoint to~$L$.

This shows that~$X$ cannot be locally path connected.
