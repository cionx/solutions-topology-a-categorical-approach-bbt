\subsection{Exercise~2.11}



\subsubsection{Subspaces}

We know that~$ℝ$ is locally compact but its subspace~$ℚ$ is not.



\subsubsection{Quotients}

We will use the following criterion for showing that a space is not compact.

\begin{proposition}
	\label{criterion for non-compactness}
	Let~$X$ be a topological space.
	Suppose there exists a sequence~$(x_n)_n$ of pairwise distinct points in~$X$ such that for every~$n ≥ 0$ the set~$\{ x_n, x_{n + 1}, x_{n + 2}, \dotsc \}$ is closed in~$X$.
	Then~$X$ is not compact.
\end{proposition}

\begin{proof}
	For the sets~$U_n ≔ X ∖ \{ x_n, x_{n + 1}, x_{n + 2}, \dotsc \}$ we have the increasing sequence~$U_0 ⊆ U_1 ⊆ U_2 ⊆ \dotsb$ of open proper subsets of~$X$.
	Therefore,~$\{ U_n \suchthat n ≥ 0 \}$ is an open cover of~$X$ that does not admit a finite subcover.
\end{proof}

%\begin{proposition}
%	\label{locally compact hausdorff spaces are regular}
%	Let~$X$ be a locally compact Hausdorff space, let~$x$ be a point in~$X$ and let~$C$ be a closed subset of~$X$, such that~$x ∉ C$.
%	Then there exist disjoint open subsets~$U$ and~$V$ of~$X$ with~$x ∈ U$ and~$C ⊆ V$.
%\end{proposition}

We consider the equivalence relation~$∼$ on~$ℝ$ that identifies all of~$ℤ$ to a single point, and leaves all other points non-identified.
Let~$X$ be the quotient space~$ℝ / {∼}$.
We show in the following that~$X$ is not locally compact.

%We observe that~$X$ is again a Hausdorff space.
%To show this, we need to check that any two distinct points~$\class{x}$ and~$\class{y}$ in~$ℝ$ can be separated by disjoint neighbourhoods.
%\begin{casedistinction}
%
%	\item
%		Suppose that both~$x$ and~$y$ are non-integral.
%		Let~$ε, δ > 0$ with
%		\begin{gather*}
%			ε < \min( x - \floor{x}, \ceil{x} - x ) \,,
%			\quad
%			δ < \min( y - \floor{y}, \ceil{y} - y ) \,,
%			\\
%			ε, δ < \abs{x - y} /2 \,.
%		\end{gather*}
%		The two open balls~$\ball(x, ε)$ and~$\ball(y, δ)$ are disjoint open subsets of~$ℝ$ that are saturated with respect to~$∼$ (because they are disjoint to~$ℤ$).
%		The images of~$\ball(x, ε)$ and~$\ball(y, δ)$ in~$X$ are the required neighbourhoods.
%
%	\item
%		Suppose that exactly one of the two points~$x$ and~$y$ is integral.
%		We may assume that~$x$ is integral, but~$y$ is not.
%
%		Let~$ε > 0$ with~$ε < \min( y - \floor{y}, \ceil{y} - y ) / 2$.
%		The sets~$U ≔ ⋃_{n ∈ ℤ} \ball(n, ε)$ and~$V ≔ \ball(y, ε)$ are then disjoint open subsets of~$ℝ$ that are saturated with respect to~$∼$.
%		The images of~$U$ and~$V$ in~$X / {∼}$ are the required neighbourhoods.
%
%	\item
%		Suppose that both~$x$ and~$y$ are integral.
%		This case cannot occur because then~$\class{x} = \class{y}$.
%
%\end{casedistinction}

Let~$π$ be the canonical quotient map from~$ℝ$ to~$X$, and let~$x$ be the point in~$X$ corresponding to~$ℤ$, i.e.,~$π^{-1} x = ℤ$.
Let~$U$ be any open neighbourhood of~$x$ and let~$K$ be any subspace of~$X$ with~$U ⊆ K$.
We show in the following that~$K$ cannot be compact.

The preimage~$π^{-1} U$ is an open subset of~$ℝ$ that contains~$0$.
It follows that
\[
	⋃_{n ∈ ℤ} {} (n - ε_n, n + ε_n) ⊆ π^{-1} U
\]
for some radii~$ε_n > 0$ with~$ε_n < 1$.
For every~$n ≥ 0$ let~$x'_n ≔ n + ε_n / 2n$, and let~$x_n$ be the image of~$x'_n$ in~$X$.
Since~$x'_n$ is contained in~$π^{-1} U$, the point~$x_n$ is contained in~$U$, and is therefore contained in~$K$.

The points~$x_n$ are pairwise distinct and non-integral, and therefore non-equivalent with respect to~$∼$.
Consequently, the sequence~$(x_n)_n$ in~$X$ consists of pairwise different points.

For every~$n ≥ 0$, the set~$\{ x'_n, x'_{n + 1}, x'_{n + 2}, \dotsc \}$ is closed in~$ℝ$ and saturated with respect to the equivalence relation~$∼$.
Consequently, the set~$\{ x_n, x_{n + 1}, x_{n + 2}, \dotsc \}$ is closed in~$X$, and therefore also closed in~$K$.

It follows from \cref{criterion for non-compactness} that~$K$ is not compact.



\subsubsection{Products}

Let us suppose that the countable product
\[
	X ≔ ℝ × ℝ × ℝ × \dotsb
\]
is locally compact.
Let~$x$ be some point in~$X$.
The point~$x$ has an open neighbourhood~$U$ that is contained in a compact subspace~$K$ of~$X$.
The open neighbourhood~$U$ contains a basic open set
\[
	V = V_1 × \dotsb × V_n × ℝ × ℝ × \dotsb
\]
around~$x$.
It follows that~$π_{n + 1} K ⊇ π_{n + 1} U ⊇ π_{n + 1} V = ℝ$ and thus~$π_{n + 1} K = ℝ$.
But~$π_{n + 1} K$ is compact because~$K$ is compact, while~$ℝ$ is not compact.
A contradiction!
