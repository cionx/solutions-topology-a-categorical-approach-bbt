\subsection{Exercise~2.10}



\subsubsection{Compact is Pseudocompact}

We have already seen that a real-valued continuous function on a compact space admits its maximum and its minimum, which entails that the function is bounded.
Therefore, compact topological spaces are also pseudocompact.



\subsubsection{Pseudocompact but not Compact}

We consider the space~$X$ whose underlying set is given by the real line~$ℝ$, and whose topology is~$\{ (-∞, a) \suchthat a ∈ [-∞, ∞] \}$.
The space~$X$ is not compact because the open cover~$\{ (-∞, n) \suchthat n ∈ ℤ \}$ does not admit a finite subcover.
But any two nonempty open subsets of~$X$ intersect, whence it follows from \cref{separating two points with map into a hausdorff space} that every continuous real-valued function on~$X$ is constant.
