\subsection{Constructions with Hausdorff Spaces}



\subsubsection{Topological Property}

\begin{lemma}
	\label{pull back Hausdorff via injective continuous maps}
	Let~$X$ and~$Y$ be two topological spaces.
	Let~$Y$ be a Hausdorff space and suppose that there exists an injective continuous map from~$X$ to~$Y$.
	Then~$X$ is again a Hausdorff space.
\end{lemma}

\begin{proof}
	Let~$f$ be an injective continuous map from~$X$ to~$Y$.

	Let~$x$ and~$x'$ be two distinct points in~$X$.
	The two points~$f x$ and~$f x'$ in~$Y$ are again distinct because~$f$ is injective.
	There exist disjoint open neighbourhoods~$V$ and~$V'$ of~$f x$ and~$f x'$ respectively.
	The preimages~$f^{-1} V$ and~$f^{-1} V'$ are disjoint open neighbourhoods of~$x$ and~$x'$ respectively.
\end{proof}

Let~$X$ and~$Y$ be two homeomorphic topological spaces.
It follows from \cref{pull back Hausdorff via injective continuous maps} that~$X$ is a Hausdorff space if and only if~$Y$ is a Hausdorff space.



\subsubsection{Not a Homotopy Invariant Property}

\begin{lemma}
	\label{all maps into indiscrete are homotopic}
	Let~$X$ be a topological space and let~$Y$ be an indiscrete topological space.
	Every two maps from~$X$ to~$Y$ are homotopic.
	\qed
\end{lemma}

It follows from \cref{all maps into indiscrete are homotopic} that all nonempty indiscrete topological spaces are homotopy equivalent.
The one-point space is a Hausdorff space, whereas the two-point indiscrete topological space is not.



\subsubsection{Subspaces of Hausdorff Spaces}

Let~$X$ be a Hausdorff space and let~$Y$ be a subspace of~$X$.
It follows from \cref{pull back Hausdorff via injective continuous maps} that~$Y$ is again a Hausdorff space.



\subsubsection{Products of Hausdorff Spaces}

Let~$(X_α)_{α ∈ A}$ be a family of topological spaces and suppose that each~$X_α$ is a Hausdorff space.
Let~$x = (x_α)_α$ and~$y = (y_α)_α$ be two distinct elements of~$∏_{α ∈ A} X_α$.
This means that there exist some index~$β$ with~$x_β ≠ y_β$.
The space~$X_β$ is a Hausdorff space, so there exists disjoint open neighbourhoods~$U$ and~$V$ of~$x_β$ and~$y_β$ respectively.
The preimages~$π_β^{-1} U$ and~$π_β^{-1} V$ are disjoint open neighbourhoods of~$x$ and~$y$ respectively.



\subsubsection{Coproducts of Hausdorff Spaces}

Let~$(X_α)_α$ be a family of Hausdorff spaces and let~$x$ and~$y$ be two elements of the coproduct~$X ≔ ∐_{α ∈ A} X_α$.
There exist indices~$α, β ∈ A$ with~$x ∈ X_α$ and~$y ∈ X_β$.
We distinguish between two cases:

\begin{casedistinction}

	\item
		Suppose that~$α ≠ β$.
		Then~$X_α$ and~$X_β$ are disjoint open neighbourhoods of~$x$ and~$y$ in~$X$ respectively.

	\item
		Suppose that~$α = β$.
		Then there exist disjoint open neighbourhoods of~$x$ and~$y$ in~$X_α$.
		These are then also disjoint open neighbourhoods of~$x$ and~$y$ in~$X$.

\end{casedistinction}



\subsubsection{Quotients of Hausdorff Spaces}

Let~$∼$ be the equivalence relation on~$ℝ$ given by
\[
	x ∼ y \iff \text{$x - y$ is rational} \,.
\]
That is, for every point~$x$ in~$ℝ$, its equivalence class is~$x + ℚ$.

Let~$U$ be nonempty saturated open subset of~$ℝ$.
The set~$U$ contains some open interval~$I = (a, b)$ with~$a < b$.
It follows that~$⋃_{x ∈ I} x + ℚ ⊆ U$ because~$U$ is saturated, but~$⋃_{x ∈ I} x + ℚ = ℝ$ since for every real number~$r$ there exists a rational number~$q$ with~$a < r + q < b$.
Therefore,~$U = ℝ$.

This shows that the only saturated open subsets of~$ℝ$ are~$∅$ and~$ℝ$.
This means that the quotient topology on~$ℝ / {∼}$ is indiscrete.
But~$ℝ / {∼}$ is uncountable since~$ℝ$ is uncountable but each equivalence class with respect to~$ℚ$ is only countable.

So while~$ℝ$ is a Hausdorff space, its quotient~$ℝ / {∼}$ is very much not a Hausdorff space.
