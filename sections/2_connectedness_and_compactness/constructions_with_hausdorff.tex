\subsection{Constructions with Hausdorff Spaces}



\subsubsection{Topological Property}

\begin{lemma}
	\label{separating two points with map into a hausdorff space}
	Let~$X$ be a topological space and let~$Y$ be a Hausdorff space.
	Let~$x$ and~$x'$ be two points in~$X$ and let~$f \colon X \to Y$ be a continuous map with~$f x ≠ f x'$.
	Then~$x$ and~$x'$ can be separated via disjoint neighbourhoods.
\end{lemma}

\begin{proof}
	There exist disjoint neighbourhoods~$V$ and~$V'$ of~$f x$ and~$f x'$ respectively.
	The preimages~$f^{-1} V$ and~$f^{-1} V'$ are disjoint neighbourhoods of~$x$ and~$x'$ respectively.
\end{proof}

\begin{proposition}
	\label{pull back Hausdorff via injective continuous maps}
	Let~$X$ and~$Y$ be two topological spaces.
	Let~$Y$ be a Hausdorff space and suppose that there exists an injective continuous map from~$X$ to~$Y$.
	Then~$X$ is again a Hausdorff space.
	\qed
\end{proposition}

Let~$X$ and~$Y$ be two homeomorphic topological spaces.
It follows from \cref{pull back Hausdorff via injective continuous maps} that~$X$ is a Hausdorff space if and only if~$Y$ is a Hausdorff space.



\subsubsection{Not a Homotopy Invariant Property}

\begin{lemma}
	\label{all maps into indiscrete are homotopic}
	Let~$X$ be a topological space and let~$Y$ be an indiscrete topological space.
	Every two maps from~$X$ to~$Y$ are homotopic.
	\qed
\end{lemma}

It follows from \cref{all maps into indiscrete are homotopic} that all nonempty indiscrete topological spaces are homotopy equivalent.
The one-point space is a Hausdorff space, whereas the two-point indiscrete topological space is not.



\subsubsection{Subspaces of Hausdorff Spaces}

Let~$X$ be a Hausdorff space and let~$Y$ be a subspace of~$X$.
It follows from \cref{pull back Hausdorff via injective continuous maps} that~$Y$ is again a Hausdorff space.



\subsubsection{Products of Hausdorff Spaces}

We will show in \nameref{exercise 2.18} that products of Hausdorff spaces are again Hausdorff spaces.



\subsubsection{Coproducts of Hausdorff Spaces}

Let~$(X_α)_α$ be a family of Hausdorff spaces and let~$x$ and~$y$ be two elements of the coproduct~$X ≔ ∐_{α ∈ A} X_α$.
There exist indices~$α, β ∈ A$ with~$x ∈ X_α$ and~$y ∈ X_β$.
We distinguish between two cases:

\begin{casedistinction}

	\item
		Suppose that~$α ≠ β$.
		Then~$X_α$ and~$X_β$ are disjoint open neighbourhoods of~$x$ and~$y$ in~$X$ respectively.

	\item
		Suppose that~$α = β$.
		Then there exist disjoint open neighbourhoods of~$x$ and~$y$ in~$X_α$.
		These are then also disjoint open neighbourhoods of~$x$ and~$y$ in~$X$.

\end{casedistinction}



\subsubsection{Quotients of Hausdorff Spaces}

We will see in \nameref{exercise 2.18} that quotients of Hausdorff spaces need not be Hausdorff spaces again.
