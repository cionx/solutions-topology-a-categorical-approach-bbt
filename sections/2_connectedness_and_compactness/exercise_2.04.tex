\subsection{Exercise~2.4}

The book is not clear on whether~$ℕ$ is supposed to include~$0$ or not.
In the original paper \autocite{golomb_connected_integers}, the topology is constructed on \enquote{the positive integers}, but allows for the arithmetic progressions~$S(a, b)$ the parameter~$b = 0$.

We will construct the topology on~$X$, which may be chosen as either one of the sets~$ℕ = \{ 0, 1, 2, \dotsc \}$ or~$ℤ_+ = \{ 1, 2, \dotsc \}$, and consider the arithmetic progressions~$S(a, b)$ with~$a ∈ ℤ_+$ and~$b ∈ X$.

(We purposefully exclude~$a = 0$.
Otherwise, each singleton set~$S(0, b) = \{ b \}$ would be open, whence the topology on~$X$ would be discrete.
But then~$X$ would be very much disconnected.)



\subsubsection{The Given Sets Form a Basis}

We denote for all~$a ∈ ℤ_+$,~$b ∈ X$ the resulting arithmetic progression by
\[
	S(a, b)
	≔ a ℕ + b
	= \{ a k + b \suchthat k ∈ ℕ \}
	= \{ b, a + b, 2a + b, \dotsc \} \,.
\]
Suppose that two arithmetic progressions~$S(a, b)$ and~$S(c, d)$ are not disjoint.
Let~$x$ be the least element of the intersection~$S(a, b) ∩ S(c, d)$.
Then
\[
	S(a, b) ∩ S(c, d) = S(\lcm(a, c), x) \,.
\]

Suppose now that~$a$ and~$b$ are coprime and that~$c$ and~$d$ are also coprime.
The element~$x$ is of the form~$ak+ b$ for some~$k ∈ ℕ$;
therefore,~$x ≡ b \bmod{a}$, and it follows that~$x$ and~$a$ are again coprime.
Similarly,~$x$ and~$c$ are again coprime.
It follows that~$x$ and~$\lcm(a, c)$ are also coprime.

These observations show that the set
\[
	\basis{B} = \{ S(a, b) \suchthat \text{$a, b ∈ ℕ$ are coprime} \}
\]
is a basis for a topology on~$X$:
for any two sets belonging to~$\basis{B}$, their intersection is either empty or again belongs to~$\basis{B}$.
Also, every element~$b$ of~$ℕ$ is contained in some set belonging to~$\basis{B}$, for example the set~$S(1, b)$.



\subsubsection{Criterion for Nonempty Intersections}

Let us consider two basic open sets~$S(a, b)$ and~$S(c, d)$.
Their intersection is nonempty if and only if
\begin{equation}
	\label{intersection of arithmetic progression is nonempty}
	\text{there exist~$k, l ∈ ℕ$ with~$k a - l c = d - b$} \,.
\end{equation}

If the condition~\eqref{intersection of arithmetic progression is nonempty} is satisfied, then it follows that~$d - b$ is divisible by the greatest common divisor of~$a$ and~$c$, which we shall denote by~$g$.

Suppose conversely that the difference~$d - b$ is divisible by~$g$.
There exists some integer~$n$ (possibly negative) with~$d - b = n g$.
We also know that there exist integers~$k'$ and~$l'$ with~$g = k' a - l' c$.
Consequently,
\[
	d - b = n g = nk' a - nl' c  \,,
\]
and thus~$nk' a + b = nl' c + d$.
This tells us that the two-sided arithmetic progressions~$ℤ a + b$ and~$ℤ c + d$ intersect.

We observe that whenever we have~$k a + b = l c + d$ for some integers~$k$ and~$l$, then we also have
\[
	(k + m c) a + b = (l + m a) c + d
\]
for every integer~$m$.
By choosing~$m$ as positive and sufficiently large, both coefficients~$k + m c$ and~$l + m a$ become natural.
We thus find that the arithmetic progressions~$S(a, b)$ and~$S(c, d)$ intersect.

Together, we find that~$S(a, b)$ and~$S(c, d)$ intersect if and only if~$\gcd(a, c)$ divides the difference~$b - d$.
Consequently,~$S(a, b)$ and~$S(c, d)$ intersect whenever~$a$ and~$c$ are coprime.



\subsubsection{The Topology is Connected}

Let~$U$ and~$V$ be two nonempty open subsets of~$X$ with~$X = U ∪ V$.

There exist arithmetic sequences~$S(a, b)$ and~$S(c, d)$ contained in~$U$ and~$V$ respectively.
It follows from the assumption~$X = U ∪ V$ that the product~$a c$ belongs to~$U$ or to~$V$.

We may assume that~$a c$ is contained in~$V$.
Then there exists an arithmetic sequence~$S(e, f)$ with~$a c ∈ S(e, f)$ and~$S(e, f) ⊆ V$.
It follows from the condition~$a c ∈ S(e, f)$ that~$a c$ is coprime to~$e$.
Therefore, both~$a$ and~$c$ are coprime to~$e$.
It follows that~$S(a, b)$ and~$S(e, f)$ intersect, and therefore~$U$ and~$V$ intersect.

We have thus shown that~$U$ and~$V$ cannot be disjoint.
