\subsection{Yoneda Embedding}



\subsubsection{Functoriality of the Yoneda Embedding}

We check that~$f_*$ is indeed a natural transformation from~$\cat{C}(\ph, X)$ to~$\cat{C}(\ph, Y)$.
We need to check that for every morphism~$g \colon W \to Z$ in~$\cat{C}$ the following diagram commutes:
\[
	\begin{tikzcd}
		\cat{C}(Z, X)
		\arrow{r}[above]{g^*}
		\arrow{d}[left]{f_*}
		&
		\cat{C}(W, X)
		\arrow{d}[right]{f_*}
		\\
		\cat{C}(Z, Y)
		\arrow{r}[above]{g^*}
		&
		\cat{C}(W, Y)
	\end{tikzcd}
\]
We have for every element~$h$ of~$\cat{C}(Z, X)$ the chain of equalities
\[
	g^* f_* h
	=
	g^* (f h)
	=
	f h g
	=
	f g^* h
	=
	f_* g^* h \,,
\]
and thus the required equality~$g^* f_* = f_* g^*$.

We note that in terms of components, given a morphism~$f \colon X \to Y$, we have~$(f_*)_Z = f_*$ for every object~$Z$ of~$\cat{C}$.
(We follow the book’s in denoting both the induced natural transformation from~$\cat{C}(\ph, X)$ to~$\cat{C}(\ph, Y)$ and also its components from~$\cat{C}(Z, X)$ to~$\cat{C}(Z, Y)$ by~$f_*$.)

We need to check that for every object~$X$, the natural transformation~$(\id_X)_*$ from~$\cat{C}(\ph, X)$ to itself agrees with~$\id_{\cat{C}(\ph, X)}$.
This holds since
\[
	( \id_{\cat{C}(\ph, X)} )_Y
	=
	\id_{\cat{C}(Y, X)}
	=
	(\id_X)_*
	=
	((\id_X)_*)_Y
\]
for every object~$Y$ of~$\cat{C}$.

Now we check that for every two morphisms~$f \colon X \to Y$ and~$g \colon Y \to Z$ in~$\cat{C}$ we have the equality of natural transformations~$(g f)_* = g_* f_*$.
This holds since
\[
	((g f)_*)_W
	=
	(g f)_*
	=
	g_* f_*
	=
	(g_*)_W (f_*)_W
	=
	( g_* f_* )_W
\]
for every object~$W$ of~$\cat{C}$.
For the second equality we used the functoriality of~$\cat{C}(W, \ph)$.



\subsubsection{Fully Faithfulness}

We now check that the Yoneda embedding is indeed fully faithful.
The Yoneda lemma, and its proof in \cref{exercise 0.6}, tells us that the map
\[
	\Nat( \cat{C}(\ph, X), \cat{C}(\ph, Y) )
	\to
	\cat{C}(X, Y) \,,
	\quad
	η \mapsto η_X \id_X
\]
is bijective.
We note that for every morphism~$f \colon X \to Y$ the preimage of~$f$ under this bijection is given by~$f_*$.
The inverse of the above bijection is thus given by
\[
	\cat{C}(X, Y) \,,
	\to
	\Nat( \cat{C}(\ph, X), \cat{C}(\ph, Y) ) \,,
	\quad
	f \mapsto f_* \,.
\]
This bijectivity tells us that the Yoneda embedding is indeed fully faithful.



\subsubsection{Yoneda’s Lemma for Covariant Functors}

Yoneda’s lemma asserts that for every covariant functor~$F \colon \cat{C} \to \Set$ and every object~$X$ of~$\cat{C}$, the map
\[
	\Nat( \cat{C}(X, -) , F ) \to F X \,,
	\quad
	η \mapsto η_X \id_X
\]
is bijective.



\subsubsection{Contravariant Yoneda Embedding}

The resulting contravariant Yoneda embedding~$\cat{C}^{\op} \to \Set^{\cat{C}}$ maps every object~$X$ of~$\cat{C}$ to the functor~$\cat{C}(X, \ph)$, and every morphism~$f \colon X \to Y$ in~$\cat{C}$ to the natural transformation~$f^* \colon \cat{C}(Y, \ph) \to \cat{C}(X, \ph)$.
