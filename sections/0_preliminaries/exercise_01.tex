\subsection{Exercise 0.1}

Let~$(\top{T}_α)_{α ∈ A}$ be a collection of topologies on the set~$X$.
Then the intersection~$\top{T} ≔ ⋂_{α ∈ A} \top{T}_α$ is again a topology on~$X$:
\begin{itemize*}

	\item
		We have $∅, X ∈ \top{T}_α$ for every index~$α ∈ A$, and therefore~$∅, X ∈ \top{T}$.

	\item
		Let~$(U_β)_{β ∈ B}$ a family of subsets of~$X$ with~$U_β ∈ \top{T}$ for every index~$β ∈ B$.
		Then~$U_β ∈ \top{T}_α$ for all indices~$α ∈ A$,~$β ∈ B$.
		It follows for every index~$α ∈ A$ that~$⋃_{β ∈ B} U_β ∈ \top{T}_α$ because~$\top{T}_α$ is a topology.
		It further follows that~$⋃_{β ∈ B} U_β ∈ \top{T}$.

	\item
		Let~$U_1, \dotsc, U_n$ be finitely many subsets of~$X$ with~$U_i ∈ \top{T}$ for every index~$i = 1, \dotsc, n$.
		Then~$U_i ∈ \top{T}_α$ for all indices~$i = 1, \dotsc, n$,~$α ∈ A$.
		Consequently,~$U_1 ∩ \dotsb ∩ U_n ∈ \top{T}_α$ for every index~$α ∈ A$ because each~$\top{T}_α$ is a topology on~$X$.
		It follows that~$U_1 ∩ \dotsb ∩ U_n ∈ \top{T}$.

\end{itemize*}

Let now~$\subbasis{S}$ be some collection of subsets of~$X$ (we do not require~$\subbasis{S}$ to cover~$X$).
We consider the set~$\mathscr{T}$ of all topologies on~$X$ that contain~$\subbasis{S}$.
The intersection~$\top{T} ≔ ⋂ \mathscr{T}$ is then a topology on~$X$.
It is, by construction, coarser than every other topology that contains~$\subbasis{S}$.
The topology~$\top{T}$ is thus the coarsest topology on~$X$ containing~$\subbasis{S}$.

Let~$\basis{B}$ be the collection of all finite intersections of sets in~$\subbasis{S}$, i.e.,
\[
	\basis{B}
	=
	\{
		S_1 ∩ \dotsb ∩ S_n
		\suchthat
		n ≥ 0,
		S_i ∈ \subbasis{S}
	\} \,.
\]
We claim that~$\basis{B}$ is a basis for a topology on~$X$.
We need to check that~$\basis{B}$ satisfies both properties from Definition~0.2.
\begin{enumerate*}[label=(\roman*)]

	\item
		We need to show that every element~$x$ of~$X$ is contained in some element~$B$ of~$\basis{B}$.
		We note that~$X$ is an element of~$\basis{B}$ because~$X = S_1 ∩ \dotsb ∩ S_n$ for~$n = 0$.
		We may therefore choose~$B$ as~$X$.
		Alternatively, if we additionally assume that~$X$ is covered by~$\subbasis{S}$ (i.e., that~$X = ⋃ \subbasis{S}$), then there exists an element~$B$ of~$\subbasis{S}$ with~$x ∈ B$.
		But the set~$B$ is also an element of~$\basis{B}$.

	\item
		Let now~$A$ and~$B$ be two sets in~$\top{B}$ and let~$x$ be an element of~$X$ with~$x ∈ A ∩ B$.
		We need to show that there exists an element~$C$ of~$\basis{B}$ with~$x ∈ C$.
		Since~$A ∩ B$ is again an element of~$\basis{B}$, we may choose~$C$ as~$A ∩ B$.

\end{enumerate*}
We have thus shown that~$\basis{B}$ is a basis for some topology on~$X$.
We need to show that this topology is precisely~$\top{T}$.

\begin{lemma}
	Let~$\basis{B}$ be a basis for a topology~$\top{T}$ on~$X$.
	Then~$\top{T}$ consists of all unions of elements of~$\basis{B}$.
\end{lemma}

\begin{proof}
	Let~$\top{T}$ be the collections of all unions of elements of~$\basis{B}$, i.e.,
	\[
		\top{T}
		=
		\Bigl\{
			⋃ \basis{C}
		\suchthat[\Big]
			\basis{C} ⊆ \basis{B}
		\Bigr\} \,.
	\]
	Every topology on~$X$ containing~$\basis{B}$ also contains~$\top{T}$.
	It therefore suffices to show that~$\top{T}$ is a topology on~$X$.

	We have~$⋃ ∅ = ∅$, so~$∅ ∈ \top{T}$.
	By property~(i) of a basis, we have~$⋃ \basis{B} = X$, so~$X ∈ \top{T}$.
	Let~$(U_α)_{α ∈ A}$ be a family of elements of~$\top{T}$.
	Each~$U_α$ is of the form~$U_α = ⋃ \basis{C}_α$ for some~$\basis{C}_α ⊆ \basis{B}$.
	The union~$\basis{C} ≔ ⋃_{α ∈ A} \basis{C}_α$ is again a subset of~$\basis{B}$, and
	\[
		⋃_{α ∈ A} U_α
		=
		⋃_{α ∈ A} ⋃ \basis{C}_α
		=
		⋃ ⋃_{α ∈ A} \basis{C}_α
		=
		⋃ \basis{C} \,.
	\]
	This shows that~$⋃_{α ∈ A} U_α$ is again an element of~$\top{T}$.

	We note that for every two elements~$B$ and~$C$ of~$\basis{B}$ their intersection~$B ∩ C$ is an element of~$\top{T}$:
	for every element~$x$ of~$B ∩ C$ there exists by property~(ii) an element~$B_x$ of~$\basis{B}$ with~$x ∈ B_x$ and~$B_x ⊆ B ∩ C$.
	It follows that~$B ∩ C = ⋃_{x ∈ B ∩ C} B_x$, whence~$B ∩ C$ is an element of~$\top{T}$.

	Let~$U$ and $V$ be two elements of~$\top{T}$.
	There exist subsets~$\basis{C}$ and~$\basis{D}$ of~$\basis{B}$ with~$U = ⋃ \basis{C}$ and~$V = ⋃ \basis{D}$.
	It follows that
	\[
		U ∩ V
		=
		\Bigl( ⋃ \basis{C} \Bigr) ∩ \Bigl( ⋃ \basis{D} \Bigr)
		=
		\Biggl( ⋃_{C ∈ \basis{C}} C \Biggr) ∩ \Biggl( ⋃_{D ∈ \basis{D}} D \Biggr)
		=
		⋃_{\substack{C ∈ \basis{C} \\ D ∈ \basis{D}}} (C ∩ D) \,.
	\]
	We have already shown that each intersection~$C ∩ D$ is an element of~$\top{T}$ and that~$\top{T}$ is closed under unions.
	Therefore,~$U ∩ V$ is an element of~$\top{T}$.
\end{proof}

Let~$\top{T}'$ be the topology on~$X$ generated by~$\basis{B}$.
We have~$\subbasis{S} ⊆ \basis{B} ⊆ \top{T}'$, whence~$\top{T}'$ is a topology on~$X$ containing~$\subbasis{S}$.
It follows that~$\top{T} ⊆ \top{T}'$.

On the other hand,~$\subbasis{S}$ is contained in~$\top{T}$, so~$\basis{B}$ is also contained in~$\top{T}$, since~$\top{T}$ is closed under finite intersections.
We find from the above lemma that~$\top{T}'$ consists of all unions of sets in~$\basis{B}$.
It follows that~$\top{T}'$ is contained in~$\top{T}$ since~$\top{T}$ is closed under unions.
Hence,~$\top{T}' ⊆ \top{T}$.

We have shown that~$\top{T} = \top{T}'$, which tells us that~$\basis{B}$ is a basis of~$\top{T}$.
