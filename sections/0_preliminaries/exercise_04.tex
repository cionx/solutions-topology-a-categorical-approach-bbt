\subsection{Exercise~0.4}



\subsubsection{Initial and terminal objects}

The trivial group is both the initial object and the terminal object in~$\Grp$.
Similarly, the zero vector space is both the initial object and the terminal object in~$\Vectk$.



\subsubsection{Products}

For every family~$(G_α)_{α ∈ A}$ of groups, the product of the~$G_α$ looks as follows:
\begin{itemize*}

	\item
		The underlying set of the product~$\prod_{α ∈ A} G_α$ is the Cartesian product of the underlying sets of the~$G_α$.

	\item
		The multiplication of~$∏_{α ∈ A} G_α$ is given by~$(g_α)_α (h_α)_α = (g_α h_α)_α$.

	\item
		For every index~$α ∈ A$ the canonical projection~$π_β$ from~$∏_{α ∈ A} G_α$ to~$G_β$ is given by~$π_β (g_α)_α  = g_β$.

\end{itemize*}
Similarly, for every family~$(V_α)_{α ∈ A}$ of vector spaces, the product of the~$V_α$ looks as follows:
\begin{itemize*}

	\item
		The underlying set of the product~$\prod_{α ∈ A} V_α$ is the Cartesian product of the underlying sets of the~$V_α$.

	\item
		The addition of~$∏_{α ∈ A} V_α$ is given by~$(v_α)_α + (w_α)_α =(v_α + w_α)_α$, and the scalar multiplication of~$∏_{α ∈ A} V_α$ is given by~$λ (v_α)_α = (λ v_α)_α$.

	\item
		For every index~$α ∈ A$, the canonical projection~$π_β$ from~$∏_{α ∈ A} V_α$ to~$V_β$ is given by~$π_β (v_α)_α  = v_β$.

\end{itemize*}



\subsubsection{Coproducts in $\Vectk$}

Given a family~$(V_α)_{α ∈ A}$ of vector spaces, the coproduct of the~$V_α$ is the direct sum~$⨁_{α ∈ A} V_α$.
For every index~$β ∈ A$, the canonical morphism~$ι_β$ from~$V_β$ to~$⨁_{α ∈ A} V_α$ is given by~$ι_β v = (v_α)_α$ with~$v_α = v$ for~$α = β$ and~$v_α = 0$ otherwise.



\subsubsection{Coproducts in $\Grp$}

Given a family~$(G_α)_{α ∈ A}$ of groups, the coproduct of the~$G_α$ is the free product~$\bigast_{α ∈ A} G_α$.
Recall that this free product can be described as follows:
\begin{itemize*}

	\item
		A \emph{word} in the~$G_α$ is a tuple of finite length~$((g_1, α_1), \dotsc, (g_n, α_n))$ where each~$α_i$ in an element of~$A$ and~$g_i$ is an element of~$G_{α_i}$ for every~$i = 1, \dotsc, n$.

	\item
		Every word~$w = ((g_1, α_1), \dotsc, (g_n, α_n))$ can be reduced by repeatedly applying the following two reduction rules:
		\begin{itemize*}

			\item
				If~$g_i$ is trivial at some position~$i$, then consider instead the word
				\[
					( (g_1, α_1), \dotsc, \widehat{ (g_i, α_i) } \dotsc, (g_n, α_n) ) \,.
				\]

			\item
				If~$α_i = α_{i + 1}$ at some position~$i$, then consider instead the word
				\[
					( (g_1, α_1), \dotsc, (g_i g_{i+1}, α_i), \dotsc, (g_n, α_n) ) \,.
				\]

		\end{itemize*}
		The word~$w$ is \emph{reduced} if it cannot be reduced any further.
		More explicitly, the word~$w$ is reduced if and only if~$g_i$ is non-trivial for every~$i = 1, \dotsc, n$ and also~$α_i ≠ α_{i + 1}$ for every~$i = 1, \dotsc, n - 1$.

	\item
		Let~$W$ be the set of all words in the~$G_α$.
		Given two words~$w = (w_1, \dotsc, w_n)$ and~$w' = (w'_1, \dotsc, w'_m)$, their concatenation is the word
		\[
			w w' = (w_1, \dotsc w_n, w'_1, \dotsc, w'_m) \,.
		\]
		Concatenation makes~$W$ into a monoid.

	\item
		Let~$∼$ be the equivalence relation on~$W$ generated by~$w ∼ w'$ whenever~$w'$ results from~$w$ by one of the reduction rules.
		The quotient~$W / {∼}$ inherits the structure of a monoid from~$W$, and~$W / {∼}$ is a group.
		The free product~$\bigast_{α ∈ A} G_α$ is defined as the group~$W / {∼}$.

	\item
		The reduced words form a set of representatives for the equivalence relation~$∼$.
		One can therefore think about the elements of~$\bigast_{α ∈ A} G_α$ as reduced words in the~$G_α$.

\end{itemize*}

For every index~$β ∈ A$, the canonical homomorphism~$ι_β$ from~$G_β$ to~$\bigast_{α ∈ A} G_α$ is given by~$ι_β g = \class{((g, β))}$.
