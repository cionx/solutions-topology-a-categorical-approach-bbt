\subsection{Exercise~0.3}
\label{exercise 0.3}



\subsubsection{a)}

Let~$f \colon X \to Y$ be a left-invertible morphism.
This means that there exists a morphism~$g \colon Y \to Z$ with~$g f = \id_X$.
Every morphism~$h \colon Z \to X$ can be retrieved from its composite~$f h$ since
\[
	g (f h) = g f h = \id_X h = h \,.
\]
The morphism~$f$ is therefore monic.

Suppose that~$f$ is a right-invertible morphism in a category~$\cat{C}$.
Then~$f$ is left-invertible in~$\cat{C}^{\op}$, and therefore monic in~$\cat{C}^{\op}$ as seen above.
This means that~$f$ is epic in~$\cat{C}$.



\subsubsection{b)}

Consider the category~$\Grp$ of groups.
The canonical quotient map~$p$ from~$ℤ$ to~$ℤ/2$ is surjective and thus epic.
But the only homomorphism of groups from~$ℤ/2$ to~$ℤ$ is the zero homomorphism, which is not a right-inverse to~$p$.
Therefore,~$p$ is not right-invertible.



\subsubsection{c)}

We defined a morphism to be invertible if and only if it is both left-invertible and right-invertible.
The given assertion is therefore true by definition.



\subsubsection{d)}

Let~$S$ be a set that consists of at least two distinct elements.
We can endow~$S$ with the discrete topology (for which every subset of~$S$ is open), resulting in a topological space~$X$.
We can also endow~$S$ with the indiscrete topology (for which the only open subsets of~$S$ are~$∅$ and~$S$), resulting in a topological space~$Y$.
The map~$\id_S$ is a continuous map from~$X$ to~$Y$, which is bijective and therefore both monic and epic.
But~$\id_S$ is not a homeomorphism because for every element~$s$ of~$S$ the set~$\{ s \}$ is open in~$X$ but not in~$Y$.



\subsubsection{e)}

We consider the category~$\Grp$ of groups and the two groups
\begin{align*}
	G &≔ ℤ/2 ⊕ ℤ/4 ⊕ ℤ/4 ⊕ ℤ/4 ⊕ \dotsb
\shortintertext{and}
	H &≔ ℤ/4 ⊕ ℤ/4 ⊕ ℤ/4 ⊕ ℤ/4 ⊕ \dotsb
\end{align*}
We have injective homomorphisms of groups
\[
	f
	\colon
	G \to H \,,
	\quad
	(x_1, x_2, \dotsc) \mapsto (0, x_1, x_2, \dotsc)
\]
and
\[
	g
	\colon
	H \to G \,,
	\quad
	(\class{x_1}, x_2, x_3, \dotsc) \mapsto (\class{2x_1}, x_2, x_3, \dotsc) \,.
\]
The homomorphisms~$f$ and~$g$ are monic since they are injective.

But the groups~$G$ and~$H$ are not isomorphic:
For every element~$y$ of~$H$ with~$2y = 0$ there exists another element~$y'$ of~$H$ with~$y = 2y'$.
But in~$G$, while the element~$x = (\class{1}, 0, 0, \dotsc)$ satisfies~$2x = 0$ there exists no element~$x'$ of~$G$ with~$x = 2x'$.

