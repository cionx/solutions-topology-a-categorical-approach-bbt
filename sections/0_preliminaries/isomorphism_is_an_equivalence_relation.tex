\subsection{\enquote{Being isomorphic} is an equivalence relation}

Let~$\cat{C}$ be a category.

For every object~$X$ of~$\cat{C}$, the identity morphism~$\id_X$ is inverse to itself, and therefore an isomorphism.
Consequently,~$X$ is isomorphic to itself.
This shows that \enquote{being isomorphic} is reflexive.

Let~$f \colon X \to Y$ be an isomorphism in~$\cat{C}$ with inverse~$f^{-1} \colon Y \to X$.
This means that~$f f^{-1} = \id_Y$ and~$f^{-1} f = \id_X$.
These equations mean that~$f^{-1}$ is an isomorphism with inverse~$f$.
Consequently, \enquote{being isomorphic} is symmetric.

Let~$f \colon X \to Y$ and~$g \colon Y \to Z$ be two isomorphisms with respective inverses~$f^{-1} \colon Y \to Z$ and~$g^{-1} \colon Z \to Y$.
Then
\[
	(g f) (f^{-1} g^{-1})
	=
	g f f^{-1} g^{-1}
	=
	g \id_Y g^{-1}
	=
	g g^{-1}
	=
	\id_Z
\]
and similarly
\[
	(f^{-1} g^{-1}) (g f)
	=
	f^{-1} g^{-1} g f
	=
	f^{-1} \id_Y f
	=
	f^{-1} f
	=
	\id_X \,.
\]
This tells us that the composite~$g f$ is again an isomorphism, with inverse given by~$f^{-1} g^{-1}$.
Consequently, \enquote{being isomorphic} is transitive.
