\subsection{Examples for Categories}



\subsubsection{The Category~$\Set$}

Composition of functions is associative.

We have for every set~$X$ the identity function~$\id_X$, which satisfies~$f \id_X = f$ for every function~$f \colon X \to Y$ and~$\id_X g = g$ for every function~$g \colon Z \to X$.



\subsubsection{The Category~$\Set_*$}

Let~$f \colon (S, s_0) \to (T, t_0)$ and~$g \colon (T, t_0) \to (U, u_0)$ be two morphisms of pointed sets.
Then
\[
	g f s_0 = g t_0 = u_0 \,.
\]
The composite function~$g f$ is therefore a morphism of pointed sets from~$(S, s_0)$ to~$(U, u_0)$.

We have for every pointed set~$(S, s_0)$ the equality~$\id_S s_0 = s_0$, where~$\id_S$ denotes the identity function.
Therefore,~$\id_S$ is a morphism of pointed sets, and it serves as the identity morphism of~$(S, s_0)$ in~$\Set_*$.



\subsubsection{The Category~$\Top$}

Let~$f \colon X \to Y$ and~$g \colon Y \to Z$ be two composable continuous maps.
Let~$U$ be an open subset of~$Z$.
The preimage~$g^{-1} U$ is open in~$Y$ because~$g$ is continuous.
Hence, the preimage~$f^{-1} g^{-1} U$ is open in~$X$ because~$f$ is continuous.
But we have~$f^{-1} g^{-1} U = (g f)^{-1} U$.
We have thus shown that the composite~$g f$ is continuous.

For every open subset~$U$ of~$X$, its preimage~$\id_X^{-1} U = U$ is open in~$X$, where~$\id_X$ denoted the identity function on the underlying set of~$X$.
Therefore, the map~$\id_X$ is continuous, and it serves as the identity morphism of~$X$ in~$\Top$.



\subsubsection{The Category~$\Top_*$}

Let~$f \colon (X, x_0) \to (Y, y_0)$ and~$g \colon (Y, y_0) \to (Z, z_0)$ be two morphisms of pointed topological spaces.
The composite~$g f$ is a continuous map from~$X$ to~$Y$ that satisfies~$g f x_0 = g y_0 = z_0$.
The composite~$g f$ is therefore a morphism from~$(X, x_0)$ to~$(Z, z_0)$.

For every pointed topological space~$(X, x_0)$, the identity map~$\id_X$ is a continuous map from~$X$ to itself with~$\id_X x_0 = x_0$.
The map~$\id_X$ is therefore a morphism from~$(X, x_0)$ to itself, and it serves at the identity morphism of~$(X, x_0)$ in~$\Top_*$.



\subsubsection{The Category~$\hTop$}

We need the following result about the compatability of homotopy with composition:

\begin{proposition}
	Let~$X$,~$Y$ and~$Z$ be topological spaces.
	\begin{enumerate}

		\item
			Let~$f_1, f_2 \colon X \to Y$ and~$g \colon Y \to Z$ be continuous maps such that~$f_1$ and~$f_2$ are homotopic.
			Then~$g f_1$ and~$g f_2$ are again homotopic.

		\item
			Let~$f \colon X \to Y$ and~$g_1, g_2 \colon Y \to Z$ be continuous maps such that~$g_1$ and~$g_2$ are homotopic.
			Then~$g_1 f$ and~$g_2 f$ are again homotopic.

	\end{enumerate}
\end{proposition}

\begin{proof}
	\leavevmode
	\begin{enumerate}

		\item
			Let~$H \colon X × \int \to Y$ be a homotopy from~$f_1$ to~$f_2$.
			The composite~$g H$ is a homotopy from~$g f_1$ to~$g f_2$.

		\item
			Let~$H \colon Y × \int \to Z$ be a homotopy from~$g_1$ to~$g_2$.
			The composite~$H (f × \id_{\int})$ is a homotopy from~$g_1 f$ to~$g_2 f$.
		\qedhere

	\end{enumerate}
\end{proof}

\begin{corollary}
	\label{homotopy respects composition}
	Let~$f_1, f_2 \colon X \to Y$ and~$g_1, g_2 \colon Y \to Z$ be continuous maps such that~$f_1$ and~$f_2$ are homotopic and also~$g_1$ and~$g_2$ are homotopic.
	Then~$g_1 f_1$ and~$g_2 f_2$ are again homotopic.
	\qed
\end{corollary}

It follows from the above \lcnamecref{homotopy respects composition} that the composition of continuous maps in~$\Top$ descends to a well-defined composition of homotopy classes of continuous maps in~$\hTop$.
The composition of homotopy classes in~$\hTop$ is associative because the original composition of continuous maps in~$\Top$ is associative.

Let~$X$ be a topological space and let~$\class{\id_X}$ be the homotopy class of the identity map of~$X$.
We have for every morphism~$\class{f} \colon X \to Y$ in~$\hTop$, which is represented by a continuous map~$f \colon X \to Y$, the equalities
\[
	\class{f} \class{\id_X}
	=
	\class{f \id_X}
	=
	\class{f} \,.
\]
We find in the same way that~$\class{\id_X} \class{g} = \class{g}$ for every morphism~$\class{g} \colon Z \to X$ in~$\hTop$.
This shows that~$\class{\id_X}$ serves as the identity morphism for~$X$ in~$\hTop$.



\subsubsection{The Category~$\Grp$}

Let~$f \colon G \to H$ and~$g \colon H \to K$ be two homomorphisms of groups.
The composite~$g f$ is again a homomorphism of groups since
\[
	g f (x_1 x_2) = g ((f x_1) (f x_2)) = (g f x_1) (g f x_2)
\]
for all~$x_1, x_2 ∈ G$.
The composition of homomorphisms of groups is associative, since the composition of functions is associative.

For every group~$G$, the identity map~$\id_G$ is a homomorphism of groups from~$G$ to~$G$ since
\[
	\id_G (x_1 x_2) = x_1 x_2 = (\id_G x_1) (\id_G x_2)
\]
for all~$x_1, x_2 ∈ G$.
The homomorphism~$\id_G$ serves as the identity morphism of~$G$ in~$\Grp$.



\subsubsection{Groups as Categories}

Let~$G$ be a group and let~$\cat{G}$ be the proposed category.
Composition of morphisms in~$\cat{G}$ is associative since the multiplication of~$G$ is associative.
The neutral element of~$G$ serves as the identity morphism of the unique object~$\bullet$ of~$\cat{G}$.



\subsubsection{Path Category of a Directed Multigraph}

Let~$Γ$ be a directed multigraph.
By a \emph{path} of length~$n$ in~$Γ$ we mean a tuple
\[
	p = (x \pmid α_1, \dotsc, α_n \pmid y)
\]
where~$x$ and~$y$ are vertices of~$Γ$ and $α_1, \dotsc, α_n$ are edges in~$Γ$ subject to the following conditions:
the source of~$α_1$ is~$x$, the target of~$α_i$ is the source of~$α_{i + 1}$ for every~$i = 1, \dotsc, n - 1$, and the target of~$α_n$ is~$y$.
The vertex~$x$ is the \emph{source} of~$p$, and the vertex~$y$ is the \emph{target} of~$p$.

Let~$p = (x \pmid α_1, \dotsc, α_n \pmid y)$ and~$q = (y \pmid β_1, \dotsc, β_m \pmid z)$ be two paths in~$Γ$ for which the target of~$p$ is the source of~$q$.
The composite of the paths~$p$ and~$q$ is the path
\[
	q p = (x \pmid α_1, \dotsc, α_n, β_1, \dotsc, β_m \pmid z) \,.
\]
Composition of paths is associative.

We have for every vertex~$x$ the path~$e_x = (x \pmid x)$ of length~$0$ at~$x$.
For every path~$p$ with source~$x$ we have~$p e_x = p$, and for every path~$q$ with target~$x$ we have~$e_x q = q$.

We have a category~$\cat{P}$ as follows:
the objects of~$\cat{P}$ are the vertices of~$Γ$;
for every two vertices~$p$ and~$q$ the morphisms in~$\cat{P}$ from~$p$ to~$q$ are the paths from~$p$ to~$q$;
composition of morphisms in~$\cat{P}$ is given by composition of paths.
For every vertex~$x$, the path~$(x \pmid x)$ serves as the identity morphism of~$x$ in~$\cat{P}$.



\subsubsection{The Opposite Category}

Let
\[
	f \colon X \to Y \,, \quad
	g \colon Y \to Z \,, \quad
	h \colon Z \to W
\]
be morphisms in~$\cat{C}^{\op}$.
This means that in~$\cat{C}$, we have
\[
	f \colon Y \to X \,, \quad
	g \colon Z \to Y \,, \quad
	h \colon W \to Z \,.
\]
In~$\cat{C}$, we have the equality
\[
	f (g h) = (f g) h \,.
\]
In~$\cat{C}^{\op}$, this becomes the equality
\[
	(h g) f = h (g f) \,.
\]
This shows that composition of morphisms in~$\cat{C}^{\op}$ is associative.

Let~$f \colon X \to Y$ be a morphism in~$\cat{C}$.
In~$\cat{C}$, we then have~$f \colon Y \to X$, and thus the equalities
\[
	\id_X f = f \,, \quad f \id_Y = f \,.
\]
In~$\cat{C}^{\op}$, these become the equalities
\[
	f \id_X = f \,, \quad \id_Y f = f \,.
\]
This shows that the identity morphisms in~$\cat{C}$ also serve as the identity morphism in~$\cat{C}^{\op}$.
